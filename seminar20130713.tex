\documentclass{beamer}
\usetheme{Frankfurt}
\title{Learning under Fear of Floating}
\author{Saki Bigio, Journal of Economic Dynamics and Control, 2009}
\date{\today}

\begin{document}
\frame{\titlepage}
\section[Outline]{}
\frame{\tableofcontents}
\section{Introduction}
\begin{frame} 
\frametitle{Introduction} 
\framesubtitle{On fear of floating, economics \'{a} l\`{a} Sargent, and calibration} 

This paper tries to explain how some developing countries overcome the ``fear of
floating" problem. 

The paper provides:
\begin{enumerate} 
\item Small description of how the``fear of floating" problem has become weaker
in the past years.
\item A simple New Keynesian small open economy model that focuses on a
developing country's central banker problem.
\item An analytical solution to the model under certain conditions.
\item Introduction of multiplier preferences (Hansen and Sargent, 2006) to see if the solution (policy) is
altered.
\item Learning properties of Central Banks in developing countries and time
varying parameters.
\item Conclusions.
\end{enumerate} 
\end{frame}

\begin{frame} 
\frametitle{Introduction} 
\begin{enumerate} 
\item \textbf{Style:} The model of this paper is based on Sargent's concept that a
model is nothing else but a probability distribution over certain outcomes.
\textbf{Results are not elegant}, to say the least. It borrows methods from a 
line of research in control theory, popularized by Sargent and Hansen (Wanting Robustness in Macroeconomics, 2007),
which is interested in decision-making under dynamic environments.
 \item \textbf{Bayesian modeling:} This paper follows the motto:``Your model is wrong. Therefore, you also need to model your model
 misspecification in your model". The dynamic system needs to capture
 misspecification in some way. A common way, if you have 2 models, is to think
 them as state matrices A and B,  in a random state evolution. The controller
 needs to make an inference about these matrices as time evolves, using numerical
methods. To see the mechanics of a Bayesian model closer, I present this
example:
 \end{enumerate}
 \end{frame}
 
\begin{frame} 
\frametitle{Introduction} 
\textbf{Bayesian model example (Cogley, Colacito and Sargent, 2007):} A decision
maker wants to maximize the following function of states $s_t$ and controls
$v_t$:
\begin{equation*}
\begin{aligned}
& \underset{}{\text{max}}
& & E_{0} $\displaystyle\sum\limits_{t=0}^{\infty} \betha^t r(s_t,v_t)$. \\
& z_{t+1} = z_{t},
& & s_{t+1} = g(s_t, v_t, z_t, \epsilon_{t+1}),
\end{aligned}
\end{equation*}
where $\epsilon_{t+1}$ is an i.i.d. vector of shocks and $z_t \in \{1,2\}$ is a
hidden state variable that indexes submodels. State variable $z_t$ is time
invariant, thus, one of the two submodels governs the data for all period. But
$z_t$ is unknown to the decision maker. He has a prior probability
$P(z=1)=\pi_{0}$. Given history $s^t = [s_t, s_{t-1},...,s_0]$, the decision
maker recursively computes $pi_t=P(z=1|s^t)$ via Bayes' law:
\end{frame}

\begin{frame} 
\frametitle{Introduction} 
\begin{equation}
\pi_{t+1} = B(\pi_t, g(s_t, v_t, z_t, \epsilon_{t+1})).
\end{equation}
E.g. In Cogley, Colacito, Hansen and Sargent (2008), one of the submodels is a
Keynesian model of a Phillips curve, while the other is a new classical model. 
The decision maker must decide while he learns、but he doesn't know $z_t$. 
His prior probability $\pi_t$ becomes a state variable in a Bellman equation
that captures his incentive to experiment:
\begin{equation}
V(s,\pi) =  \underset{v}{\text{max}} \{ r(s,v) + E_z[E_{s^*,\pi^*}( \beta V(s^*,\pi^*) |s,v,\pi,z) | s,v,\pi]\}, s.t.
\end{equation}
\begin{equation}
s^{*} = g(s,v,z,\epsilon^*),
\end{equation}
\begin{equation}
\pi^{*} = B(\pi,g(s,v,z,\epsilon^*)),
\end{equation}
\end{frame}

\begin{frame} 
where the asterisks denote next-period values, $E_z$ integration with
respect to the distribution of the hidden state $z$ that indexes submodels, and
$E_{s^*,\pi^*}$ denotes integration with respect to the joint distribution of
$(s^*,\pi^*)$ conditional on $(s,v,\pi,z)$. The Bellman equation expresses the motivation
a decision maker has to experiment (to take into consideration how his decision
affects future values of $\pi^*$). This equation shows 2 possible types of 
misspecification: distribution of $(s^*, \pi^*)$ conditional on $(s,v,\pi,z)$
and misspecification of the probability $\pi$ over submodels $z$.
\newline
\newline
Recent developments in macroeconomics make it possible to get a decision rule that is
robust to both misspecifications. Examples include using  the method of risk-sensitivity operators (Hansen and Sargent, 2007)
or applying a Kalman filter.
\end{frame}

\section{Fear of Floating}
 \begin{frame} 
\frametitle{Fear of Floating} 
\framesubtitle{Cross-country evidence} 
In developing economies, Central Bankers think of exchange rates as having two
potential effects. High unexpected depreciations
\begin{enumerate} 
\item Standard economics textbook effect: Are good for agents who make money in dollars (exporting sector). Under
nominal rigidities, a sudden shift in value of nominal exchange rate 
translates to a sudden shift in real exchange rate and produces positive wealth effects.
\item Balance sheet effect: Are bad for agents that have a considerable
proportion of their assets in foreign currency (e.g. partial dollarization in undeveloped
financial markets, where firms hold assets in domestic currency while liabilities in
US Dollars, with no derivatives).
\end{enumerate} 
\end{frame}

\begin{frame} 
\frametitle{Fear of Floating} 
\textbf{Fact:} In recent years, a large number of developing countries
seem to have moved towards more flexible exchange rates.
\newline
\newline 
\textbf{Hypothesis:}  Maybe Central Bankers learned about these effects.
\newline
\newline
\textbf{Method:} Model the problem of the Central Bank as having uncertainty
over a parameter $\theta$ that englobes both effects. 2 sub-models. Each sub-model delivers opposite effects.
The Central Banker fears missing the true model. A Bayesian Central Banker
will react according prior beliefs about either model and
update them via Bayes rule. The paper also considers the case when
the (optimal) Central Bank behaves according to a policy consistent
with fear of misspecification.  
\end{frame}

\begin{frame} 
\textbf{Hypothesis:} Developing Central Banks fixed echanged raes in the
nineties and took them more than 10 years to learn they were wrong.
Slow learning was caused by a fear of floating policy. Because of uncertainty,
no meaningful exchange rate movement is allowed. This makes it difficult to
detect the real effect and correct model. 
\newline
\newline
\textbf{TODO:} Calibration of this paper model shows it takes 5 years for
Central Bankers to learn. This is faster than what cross-country data suggests (10
years). Maybe there was an exogenous parameter change from balance-sheet model
to textbook model in the 2000s?
\end{frame}


\begin{frame} 
\frametitle{Learning under Fear} 
\framesubtitle{Evidence} 
\includegraphics[width=0.88\textwidth]{00a.pdf}
\end{frame}

\section{Model}
\begin{frame} 
\frametitle{A Standard Small Open Economy Model: The Central Banker problem} 
\framesubtitle{Assumptions} 
Let the Phillips curve be represented as:
\begin{equation}
\pi_t = \beta E_t [\pi_{t+1}] + \gamma y_t + \varepsilon_{\pi,t},
\end{equation}
where $\pi_t$ is the inflation rate, $y_t$ in the output gap $\frac{GDP_{actual}-GDP_{potential}}{GDP_{potential}}$
and  $\beta$ is the period discount factor. Notice that for $f: \mathbb{R} \rightarrow
\mathbb{R}$,
\begin{equation}
\pi_t = f(E_t[\pi_{t+1}], y_t),
\end{equation}
where  $\varepsilon_{\pi,t}$ is a cost-push shock.
\end{frame}

\begin{frame} 
Let the aggregate demand equation:
\begin{equation}
y_t = E_t[y_{t+1}] - \chi(i_t - E_t[\pi_{t+1}] - r_t^n) + \theta
E_t[\Delta s_{t+1}] + \varepsilon_{y,t}.
\end{equation}
Intuitively, the output gap depends on expected output for next period, the
gap between real interest rates and its natural level, expected
nominal depreciation $\theta
E_t[\Delta s_{t+1}] $ and a demand shock. 
\newline
\newline
The textbook model has positive $\theta$.
The balance sheet model has negative $\theta$.
Also notice that the expected nominal exchange rate depreciation comes from an interest rate parity equation:
\begin{equation}
E_t[\Delta s_{t+1}] = i_t - i_t^* - \varepsilon_{et},
\end{equation}
which is a non-arbitrage condition between domestic $i_t$ and foreign $i_t^*$. $\varepsilon_{et}$
is a financial shock. 
\newline
\newline
$r_t$ follows an autoregressive process:
\begin{equation}
r_t^n = \rho_n r_t^n + \varepsilon_t^r.
\end{equation}
\end{frame}

\section{Optimal Bayesian Policy}
\begin{frame} 
\frametitle{Optimal Bayesian Policy} 
\end{frame}


\section{Optimal Bayesian Policy}
\begin{frame} 
\frametitle{Optimal Bayesian Policy} 
\end{frame}


\section{Robust Policy}
\begin{frame} 
\frametitle{Robust Policy} 
\end{frame}


\section{Model Dynamics and Learning}
\begin{frame} 
\frametitle{Effects of increase in rate of monetary expansion} 
\includegraphics[width=0.85\textwidth]{phase5.pdf}
\end{frame}

\begin{frame} 
\frametitle{Final Remarks} 

\end{frame} 
\end{document}
