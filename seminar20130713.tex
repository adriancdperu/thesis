% Presentation 2013 07 13
% TODO: 軽く論文もう一度読み流す、state matrixの解け方を覚える、Section 5を暗記

\documentclass{beamer}
\usetheme{Frankfurt}
\title{Learning under Fear of Floating}
\author{Osaka, DMW}
\subtitle{Saki Bigio, Journal of Economic Dynamics and Control, 2009}
\date{\today}

\begin{document}
\frame{\titlepage}
\section[Outline]{}
\frame{\tableofcontents}
\section{Introduction}
\begin{frame} 
\frametitle{Introduction} 
\framesubtitle{On fear of floating, economics \'{a} l\`{a} Sargent, and calibration} 
This paper tries to explain how some developing countries overcome the ``fear of
floating" problem. 
The paper provides:
\begin{enumerate} 
\item Small description of how the``fear of floating" problem has become weaker
in the past years.
\item A simple New Keynesian small open economy model that focuses on a
developing country's central banker problem.
\item An analytical solution to the model under certain conditions.
\item Introduction of multiplier preferences (Hansen and Sargent, 2006) to see if the solution (policy) is
altered.
\item Learning properties of Central Banks in developing countries and time
varying parameters.
\item Conclusions.
\end{enumerate} 
\end{frame}

\begin{frame} 
\frametitle{Introduction} 
\begin{enumerate} 
\item \textbf{Style:} The model of this paper is based on Sargent's concept that a
model is nothing else but a probability distribution over certain outcomes.
\textbf{Results are not elegant}, to say the least. It borrows methods from a 
line of research in control theory, popularized by Sargent and Hansen (Wanting Robustness in Macroeconomics, 2007),
which is interested in decision-making under dynamic environments.
 \item \textbf{Bayesian modeling:} This paper follows the motto:``Your model is wrong. Therefore, you also need to model your model
 misspecification in your model". The dynamic system needs to capture
 misspecification in some way. A common way, if you have 2 models, is to think
 them as state matrices A and B,  in a random state evolution. The controller
 needs to make an inference about these matrices as time evolves, using numerical
methods. To see the mechanics of a Bayesian model closer, I present this
example:
 \end{enumerate}
 \end{frame}
 
\begin{frame} 
\frametitle{Introduction} 
\textbf{Bayesian model example (Cogley, Colacito and Sargent, 2007):} A decision
maker wants to maximize the following function of states $s_t$ and controls
$v_t$:
\begin{equation*}
\begin{aligned}
& \underset{}{\text{max}}
& & E_{0} $\displaystyle\sum\limits_{t=0}^{\infty} \betha^t r(s_t,v_t)$. \\
& z_{t+1} = z_{t},
& & s_{t+1} = g(s_t, v_t, z_t, \epsilon_{t+1}),
\end{aligned}
\end{equation*}
where $\epsilon_{t+1}$ is an i.i.d. vector of shocks and $z_t \in \{1,2\}$ is a
hidden state variable that indexes submodels. State variable $z_t$ is time
invariant, thus, one of the two submodels governs the data for all period. But
$z_t$ is unknown to the decision maker. He has a prior probability
$P(z=1)=\pi_{0}$. Given history $s^t = [s_t, s_{t-1},...,s_0]$, the decision
maker recursively computes $pi_t=P(z=1|s^t)$ via Bayes' law:
\end{frame}

\begin{frame} 
\frametitle{Introduction} 
\begin{equation}
\pi_{t+1} = B(\pi_t, g(s_t, v_t, z_t, \epsilon_{t+1})).
\end{equation}
E.g. In Cogley, Colacito, Hansen and Sargent (2008), one of the submodels is a
Keynesian model of a Phillips curve, while the other is a new classical model. 
The decision maker must decide while he learns、but he doesn't know $z_t$. 
His prior probability $\pi_t$ becomes a state variable in a Bellman equation
that captures his incentive to experiment:
\begin{equation}
V(s,\pi) =  \underset{v}{\text{max}} \{ r(s,v) + E_z[E_{s^*,\pi^*}( \beta V(s^*,\pi^*) |s,v,\pi,z) | s,v,\pi]\}, s.t.
\end{equation}
\begin{equation}
s^{*} = g(s,v,z,\epsilon^*),
\end{equation}
\begin{equation}
\pi^{*} = B(\pi,g(s,v,z,\epsilon^*)),
\end{equation}
\end{frame}

\begin{frame} 
where the asterisks denote next-period values, $E_z$ integration with
respect to the distribution of the hidden state $z$ that indexes submodels, and
$E_{s^*,\pi^*}$ denotes integration with respect to the joint distribution of
$(s^*,\pi^*)$ conditional on $(s,v,\pi,z)$. The Bellman equation expresses the motivation
a decision maker has to experiment (to take into consideration how his decision
affects future values of $\pi^*$). This equation shows 2 possible types of 
misspecification: distribution of $(s^*, \pi^*)$ conditional on $(s,v,\pi,z)$
and misspecification of the probability $\pi$ over submodels $z$.
\newline
\newline
Recent developments in macroeconomics make it possible to get a decision rule that is
robust to both misspecifications. Examples include using  the method of risk-sensitivity operators (Hansen and Sargent, 2007)
or applying a Kalman filter.
\end{frame}

\section{Fear of Floating}
 \begin{frame} 
\frametitle{Fear of Floating} 
\framesubtitle{Cross-country evidence} 
In developing economies, Central Bankers think of exchange rates as having two
potential effects. High unexpected depreciations
\begin{enumerate} 
\item Standard economics textbook effect: Are good for agents who make money in dollars (exporting sector). Under
nominal rigidities, a sudden shift in value of nominal exchange rate 
translates to a sudden shift in real exchange rate and produces positive wealth effects.
\item Balance sheet effect: Are bad for agents that have a considerable
proportion of their assets in foreign currency (e.g. partial dollarization in undeveloped
financial markets, where firms hold assets in domestic currency while liabilities in
US Dollars, with no derivatives).
\end{enumerate} 
\end{frame}

\begin{frame} 
\frametitle{Fear of Floating} 
\textbf{Fact:} In recent years, a large number of developing countries
seem to have moved towards more flexible exchange rates.
\newline
\newline 
\textbf{Hypothesis:}  Maybe Central Bankers learned about these effects.
\newline
\newline
\textbf{Method:} Model the problem of the Central Bank as having uncertainty
over a parameter $\theta$ that englobes both effects. 2 sub-models. Each sub-model delivers opposite effects.
The Central Banker fears missing the true model. A Bayesian Central Banker
will react according prior beliefs about either model and
update them via Bayes rule. The paper also considers the case when
the (optimal) Central Bank behaves according to a policy consistent
with fear of misspecification.  
\end{frame}

\begin{frame} 
\textbf{Hypothesis:} Developing Central Banks fixed echanged raes in the
nineties and took them more than 10 years to learn they were wrong.
Slow learning was caused by a fear of floating policy. Because of uncertainty,
no meaningful exchange rate movement is allowed. This makes it difficult to
detect the real effect and correct model. 
\newline
\newline
\textbf{TODO:} Calibration of this paper model shows it takes 5 years for
Central Bankers to learn. This is faster than what cross-country data suggests (10
years). Maybe there was an exogenous parameter change from balance-sheet model
to textbook model in the 2000s?
\end{frame}


\begin{frame} 
\frametitle{Learning under Fear} 
\framesubtitle{Evidence} 
\includegraphics[width=0.88\textwidth]{00a.pdf}
\end{frame}

\section{Model}
\begin{frame} 
\frametitle{A Standard Small Open Economy Model: The Central Banker problem} 
\framesubtitle{Assumptions} 
Let the Phillips curve be represented as:
\begin{equation}
\pi_t = \beta E_t [\pi_{t+1}] + \gamma y_t + \varepsilon_{\pi,t},
\end{equation}
where $\pi_t$ is the inflation rate, $y_t$ in the output gap $\frac{GDP_{actual}-GDP_{potential}}{GDP_{potential}}$
and  $\beta$ is the period discount factor. Notice that for $f: \mathbb{R} \rightarrow
\mathbb{R}$,
\begin{equation}
\pi_t = f(E_t[\pi_{t+1}], y_t),
\end{equation}
where  $\varepsilon_{\pi,t}$ is a cost-push shock.
\end{frame}

\begin{frame} 
Let the aggregate demand equation:
\begin{equation}
y_t = E_t[y_{t+1}] - \chi(i_t - E_t[\pi_{t+1}] - r_t^n) + \theta
E_t[\Delta s_{t+1}] + \varepsilon_{y,t}.
\end{equation}
Intuitively, the output gap depends on expected output for next period, the
gap between real interest rates and its natural level, expected
nominal depreciation $\theta
E_t[\Delta s_{t+1}] $ and a demand shock. 
\newline
\newline
The textbook model has positive $\theta$.
The balance sheet model has negative $\theta$.
Also notice that the expected nominal exchange rate depreciation comes from an interest rate parity equation:
\begin{equation}
E_t[\Delta s_{t+1}] = i_t - i_t^* - \varepsilon_{et},
\end{equation}
which is a non-arbitrage condition between domestic $i_t$ and foreign $i_t^*$. $\varepsilon_{et}$
is a financial shock. Also, assume that $r_t$ follows an autoregressive process:
\begin{equation}
r_t^n = \rho_n r_t^n + \varepsilon_t^r.
\end{equation}
\end{frame}

\begin{frame} 
Following Clarida et al. (A simple framework for international monetary
policy analysis, Journal of Monetary Economics, 2002), the Central Banker
problem can be expressed as:
\begin{equation*}
\begin{aligned}
& \underset{}{\text{minimize}}
& & L_\tau = E_\tau \left [ \sum_{t=\tau}^\infty \beta^{t-\tau} (\pi_t^2 +  w y_t^2 ) \right ] \\
& \text{subject to}
\end{aligned}
\end{equation*}
\begin{equation*}
\pi_t = \beta E_t [\pi_{t+1}] + \gamma y_t + \varepsilon_{\pi,t},
\end{equation*}
\begin{equation*}
y_t = E_t[y_{t+1}] - \chi(i_t - E_t[\pi_{t+1}] - r_t^n) + \theta
E_t[\Delta s_{t+1}] + \varepsilon_{y,t},
\end{equation*}
\begin{equation*}
E_t[\Delta s_{t+1}] = i_t - i_t^* - \varepsilon_{et} ,
\end{equation*}
\begin{equation*}
r_t^n = \rho_n r_t^n + \varepsilon_t^r.
\end{equation*}
Optimal policies with certainty will be the limit cases of
the Bayesian policies described in the next section. Optimal discretionary
policy (without commitment) can be obtained as a function of parameters and
observed shocks. Credible policies under commitment remains an unsolved
question.
\end{frame}

\begin{frame} 
Equations (5), (6) and (7) can be reduced to a simple two-equation systems by
substitution. Replacing (7) in (6) and setting $\mu_{y,t} = \chi r^n_t -\theta i^*_t - \theta \varepsilon_{et}+\varepsilon_{y,t}$:
\begin{equation}
\pi_t = \beta E_t [\pi_{t+1}] + \gamma y_t + \varepsilon_{\pi,t}
\end{equation}
\begin{equation}
y_t = E_t [y_{t+1}] - (\chi - \theta) i_t + \chi E_t [\pi_{t+1}] + \mu_{y,t}.
\end{equation}
Equation 5 (a.k.a Phillips curve) and equation 7 (aggregate demand equation, a.k.a IS equation)
can be used to infer the policy rate that implements the solution:
\begin{equation}
\mathcal{L}_\tau = E_\tau \left [ \sum_{t=\tau}^\infty \beta^{t-\tau} \left \{\pi_t^2 + w y_t^2 + \lambda_t (\beta \pi_{t+1} + \gamma y_t + \varepsilon_{\pi,t} - \pi_t) \right \} \right
].
\end{equation}
A discretionary solution implies that the following first-order conditions with
respect to $E_t [\pi_t]$ and $E_t [y_t]$ hold $\forall t$:
\end{frame}

\begin{frame} 
\begin{equation}
2 E_t [\pi_t] = E_t[\lambda_t],
\end{equation}
\begin{equation}
2 w E_t [y_t] + \gamma E_t [\lambda_t] = 0.
\end{equation}
Assume the following relation between the 2 variables:
\begin{equation}
-\frac{\gamma}{w} E_t [\pi_t] = E_t [y_t].
\end{equation}
Replace the above condition in the Phillips curve, to get:
\begin{equation}
\left ( 1+\frac{\gamma^2}{w} \right ) E_t [\pi_t] = \beta E_\tau [\pi_{t+1} ] + \rho_\pi
\varepsilon_{\pi,t-1}.
\end{equation}
\textbf{How to solve this equation?} Guess a solution for the
expectations operator
\begin{equation}
E_t[\pi_t] = \Lambda \varepsilon_{\pi,t-1}
\end{equation}
where $\Lamba$ is a parameter to be determined. This implies:
\begin{equation}
E_t [\pi_{t+1} = \lambda \rho_\pi \varepsilon_{\pi,t-1}.
\end{equation}
Substitute this guess back in the Phillips curve, and solve for the unknown
parameter:
\end{frame}

\begin{frame} 
\begin{equation}
\Lambda = \frac{\rho_\pi}{1+\frac{\gamma^2}{w} - \beta \rho_\pi}.
\end{equation}
Remember equation (11)? Move the interest rate to the left hand side:
\begin{equation}
i_t = \frac{1}{(\chi-\theta)} \{E_t[y_{t+1}] - E_t[y_t] + \chi E_t [\pi_{t+1}] + E_t[\mu_{y,t}]
\}.
\end{equation}
Now replace the initial guess into the interest rate function above us:
\begin{equation}
i_t = \frac{1}{(\chi-\theta)} \left \{ \left ( \frac{\gamma}{w} (\frac{1}{\rho_\pi} - 1) + \chi \right ) \Lambda \rho_\pi \varepsilon_{\pi,\tau-1} + E_t [\mu_{y,t}] \right \}
\end{equation}
\begin{theorem}
In a case of perfect certainty, the optimal policy for this model exists and has
the shape:
\begin{equation}
i_t = \frac{1}{(\chi-\theta)} \left \{ \left ( \frac{\gamma}{w} (\frac{1}{\rho_\pi} - 1) + \chi \right ) \Lambda \rho_\pi \varepsilon_{\pi,\tau-1} + E_t [\mu_{y,t}] \right \}
\end{equation}
\end{theorem}
\end{frame}

\begin{frame} 
Now, if $\rho_\pi_t \to 0$:
\begin{equation}
i_t \to \frac{1}{2} \frac{1}{(\chi - \theta)} E_t [\mu_{y,t}].
\end{equation}
which looks much prettier and should represent a simple rule that attains zero
inflation by reacting to observed shocks to the output gap.
\newline
\newline
Assume that $\varepsilon_{e,t}$ and $\varepsilon_{i,t}$ are observable and let
the solution to the expectation of exogenous shocks be:
\begin{equation}
E_t[\mu_{y,t}] = \chi (\rho_r \varepsilon_{r,t-1} ) - \theta \varepsilon_{i,t} - \theta \varepsilon_{e,t} + \rho_{y,t}
\varepsilon_{y,t-1}.
\end{equation}
Substitute this in equation (11) and solve for output:
\begin{equation}
y_t = E_t [y_{t+1}] - \left ( \frac{\gamma}{w} ( \frac{1}{\rho_\pi} - 1) + \chi \right ) \Lambda \rho_\pi \varepsilon_{\pi,\tau-1} + \chi E_t[\pi_{t+1}] + \mu_{y,t} - E_t[\mu_{y,t}]
\end{equation}
Using the initial guess, this equation can be simplified to:
\begin{equation}
y_t = -\frac{\gamma}{w} \Lambda \varepsilon_{\pi,\tau-1} + \mu_{y,t} -
E_t[\mu_{y,t}].
\end{equation}
\end{frame}

\begin{frame} 
This is somewhat a confirmation that the guess was correct. Since 
$E_t[\mu_{y,t}] = \mu_{y,t} - \chi \nu_{r,t} - \nu_{y,t}$:
\begin{equation}
y_t = -\frac{\gamma}{w} \Lambda \varepsilon_{\pi,\tau-1} + \chi \nu_{r,t} +
\nu_{y,t}.
\end{equation}
But from our first equation (5),  $\pi_t = \beta E_t [\pi_{t+1}] + \gamma y_t +
\varepsilon_{\pi,t}$, so:
\begin{equation}
\begin{align}
\pi_t &= \beta \Lambda \rho_\pi \varepsilon_{\pi,t-1} + \gamma \left ( -\frac{\gamma}{w} \Lambda \varepsilon_{\pi,\tau-1} + \chi \nu_{r,t} + \nu_{y,t} \right ) + \varepsilon_{\pi,t} \\
&= \Lambda (\beta \rho_\pi - \frac{\gamma^2}{w}) \varepsilon_{\pi,t-1} + \gamma \chi \nu_{r,t} + \gamma \nu_{y,t} + \rho_p \varepsilon_{\pi,t-1} + \nu_{\pi,t}
\end{align}
\end{equation}
Summarizing:
\begin{equation}
\pi_t = ((\Lambda \beta + 1) \rho_\pi - \frac{\gamma^2}{w} \Lambda) \varepsilon_{\pi,t-1} + \gamma \chi \nu_{r,t} + \gamma \nu_{y,t} +
\nu_{\pi,t}.
\end{equation}
Recall the uncovered interest rate parity equation (8) that states $E_t[\Delta s_{t+1}] = i_t - i_t^* -
\varepsilon_{et}$? We used it to obtain the value of the exchange rate:
\begin{equation}
E_t [\Delta s_{t+1}] = i_t - i_t^* - \varepsilon_{et}.
\end{equation}
\end{frame}

\begin{frame} 
\textbf{Steady state:} Can be obtained directly from the theorem equation of
optimal interest rate. In the absence of shocks, consider only
$\bar{\mu}_{y,t}$, $\bar{\iota}$ and $\bar{\pi}$. Steady state for the stochastic
process is:
\begin{equation}
\bar{\mu}_y = \chi \bar{r} + \theta \bar{\iota}^*.
\end{equation}
Then, for the interest rate:
\begin{equation}
\bar{\iota} = \frac{1}{(\chi-\theta)}[\bar{\mu}_y].
\end{equation}
From the steady state version of Phillips curve and output gap condition:
\begin{equation}
0 = - (\chi - \theta) \bar{\iota} + \chi \bar{\pi} + \bar{\mu}_y,
\end{equation}
which implies:
\begin{equation}
\bar{\pi} = 0.
\end{equation}
So, in absence of shocks, the policy gives the ``highest lowest cost possible".
\end{frame}

\begin{frame} 
\frametitle{State-Space representation} 
\textbf{State-Space representation:} Writing the dynamic model in state-space representation form can be useful for parameter estimation, inference and Kalman filter usage.
The exogenous state block comes from the vector equation $S_t = A S_{t-1} + B
w_t$, which in matrix notation yields:
{\tiny
\begin{equation}
\begin{bmatrix}
\varepsilon_{\pi,t} \\
\varepsilon_{\pi,t-1} \\
\varepsilon_{y,t} \\
\varepsilon_{y,t-1} \\
r_t^n \\
r_{t-1}^n \\
i_t^* \\
\varepsilon_{e,t} \\
1
\end{bmatrix}
=
\begin{bmatrix}
\rho_\pi & 0 & 0 & 0 & 0 & 0 & 0 & 0 & 0 \\
1 & 0 & 0 & 0 & 0 & 0 & 0 & 0 & 0 \\
0 & 0 & \rho_y & 0 & 0 & 0 & 0 & 0 & 0 \\
0 & 0 & 1 & 0 & 0 & 0 & 0 & 0 & 0 \\
0 & 0 & 0 & 0 & \rho_r & 0 & 0 & 0 & r^n \\
0 & 0 & 0 & 0 & 1 & 0 & 0 & 0 & 0 \\
0 & 0 & 0 & 0 & 0 & 0 & \rho_i & 0 & r^n \\
0 & 0 & 0 & 0 & 0 & 0 & 0 & \rho_e & 0 \\
0 & 0 & 0 & 0 & 0 & 0 & 0 & 0 & 1 \\
\end{bmatrix}
\begin{bmatrix}
\varepsilon_{\pi,t-1} \\
\varepsilon_{\pi,t-2} \\
\varepsilon_{y,t-1} \\
\varepsilon_{y,t-2} \\
r_{t-1}^n \\
r_{t-2}^n \\
i_{t-1}^* \\
\varepsilon_{e,t-1} \\
1
\end{bmatrix}
+
\begin{bmatrix}
\sigma_\pi & 0 & 0 & 0 & 0 \\
0 & 0 & 0 & 0 & 0 \\
0 & \sigma_y & 0 & 0 & 0 \\
0 & 0 & 0 & 0 & 0 \\
0 & 0 & \sigma_r & 0 & 0 \\
0 & 0 & 0 & 0 & 0 \\
0 & 0 & 0 & \sigma_i & 0 \\
0 & 0 & 0 & 0 & \sigma_e \\
0 & 0 & 0 & 0 & 0 
\end{bmatrix}
\begin{matrix}
\nu_{\pi,t} \\
\nu_{y,t} \\
\nu_{r,t} \\
\nu_{i,t+1} \\
\nu_{s,t+1} 
\end{matrix}
\end{equation}
}
The convention:
\begin{equation}
i^*_t = \varepsilon_{i,t}
\end{equation}
is adopted. Now, the observable vector-equation for this model is $Z_t = C S_t + D
u_t$.
\end{frame}

\begin{frame} 
 \frametitle{State-Space representation} 
Furthermore, the following parameters auxiliaries are used to find:
\newline
\newline
\textbf{Policy Reaction:} The following set of auxiliary parameters is defined.
\begin{equation}
F_1 = \frac{1}{(\chi-\theta)} \left (\frac{\gamma}{w} (\frac{1}{\rho_\pi} - 1) + \chi \right ) \Lambda \rho_\pi
\end{equation}
\begin{equation}
F_2 = \frac{1}{(\chi-\theta)} p_{y,t}
\end{equation}
\begin{equation}
F_3 = \frac{1}{(\chi-\theta)} \chi \rho_r
\end{equation}
\begin{equation}
F_4 = -\frac{1}{(\chi-\theta)} \theta
\end{equation}
\begin{equation}
F_5 = -\frac{1}{(\chi-\theta)} \theta
\end{equation}
\begin{equation}
i_t = F_1 \varepsilon_{\pi,t-1} + F_2 \varepsilon_{y,t-1} + F_3 \varepsilon_{r,t-1} + F_4 \varepsilon_{i,t} + F_5 \varepsilon_{e,t}
\end{equation}
\end{frame}

\begin{frame} 
\frametitle{State-Space representation} 
\textbf{Phillips Curve:} From equation (29):
\tiny
\begin{equation}
\pi_t = ((\Lambda \beta + 1) \rho_\pi - \frac{\gamma^2}{w} \Lambda) \varepsilon_{pi,t-1} + \gamma \chi (r_t^n - \rho_r r_{t-1}^n) + \gamma (\varepsilon_{y,t} - \rho_y \varepsilon_{y,t-1}) + (\varepsilon_{\pi,t} - \rho_\pi \varepsilon_{\pi,t-1})
\end{equation}
\normalsize
The following auxiliaries are used:
\begin{equation}
G_1 = 1
\end{equation}
\begin{equation}
G_2 = \Lambda (\beta - \frac{\gamma^2}{w})
\end{equation}
\begin{equation}
G_3 = \gamma
\end{equation}
\begin{equation}
G_4 = -\gamma \rho_y
\end{equation}
\begin{equation}
G_5 = \gamma \chi
\end{equation}
\begin{equation}
G_6 = -\gamma \chi \rho_r
\end{equation}
\end{frame}

\begin{frame} 
\frametitle{State-Space representation} 
\textbf{Output Gap:} From equation 26,
\begin{equation}
y_t = - \frac{\gamma}{w} \Lambda \varepsilon_{\pi,\tau-1} + \chi (r_t^n - \rho_r r_{t-1}^n ) + \varepsilon_{y,t} - \rho_y \varepsilon_{y,t-1}
\end{equation}
Define the following auxiliary variables: 
\begin{equation}
H_1 = -\frac{\gamma}{w} \Lambda
\end{equation}
\begin{equation}
H_2 = 1
\end{equation}
\begin{equation}
H_3 = -\rho_y
\end{equation}
\begin{equation}
H_4 = \chi
\end{equation}
\begin{equation}
H_5 = \chi \rho_r
\end{equation}
\end{frame}

\begin{frame} 
\frametitle{State-Space representation} 
\textbf{Nominal depreciation and final matrix form:} The final formal should look like:
\begin{equation}
Z_t = C S_t
\end{equation}
Equation (30) is helpful in this case. It can be written as a function
of the exogenous processes that affect the policy instrument and the
observable shocks that affect these equations directly:
{\tiny
\begin{equation}
\begin{bmatrix}
i_t \\
\pi_t \\
y_t \\
E[\Delta s_{t+1}]
\end{bmatrix}
=
\begin{bmatrix}
0 & F_1 & 0 & F_2 & 0 & F_3 & F_4 & F_5 & 0 \\
G_1 & G_2 & G_3 & G_4 & G_5 & G_6 & 0 & 0 & 0 \\
0 & H_1 & H_2 & H_3 & H_4 & H_5 & 0 & 0 & 0 \\
0 & F_1 & 0 & F_2 & 0 & F_3 & F_4-1 & F_5-1 & 0 \\
\end{bmatrix}
\begin{bmatrix}
\varepsilon_{\pi,t} \\
\varepsilon_{\pi,t-1} \\
\varepsilon_{y,t} \\
\varepsilon_{\pi,t-1} \\
r_t^n \\
r_{t-1}^n \\
i_t^* \\
\varepsilon_{e,t} \\
1 
\end{bmatrix}.
\end{equation}
}
\end{frame}

\section{Optimal Bayesian Policy}
\begin{frame} 
\frametitle{Optimal Bayesian Policy} 
From now on, the Central Banker will ignore whether data is generated by
model A or B (that is, whether $\theta$ is positive or negative). Under such condition,
the Central Banker will act in accordance to a model prior probability assigned to each
model, and maximize expected utility.  
\newline
\newline
\textbf{Knowledge assumptions:} Assume $\theta$ has an unknown value to both
the Central Banker and agents within the model. 
Central Bankers won't take into account the
effect that their actions will have on their ability to learn.  
Also, assume ``the common prior assumption": it is possible to represent any
uncertain environment by some state space and a common prior probability
distribution. Central Bankers and agents may have different information about
the true state of the world, and thus different posterior beliefs, but those
posteriors are derived by updating the common prior. Uncertainty is common
knowledge.
\end{frame}

\begin{frame}
\textbf{The Bayesian Central Bank's Problem:} Since the Banker is uncertain
about the actual model driving the economy, he assigns a probability $p_t$ to
model A and probability $1-p_t$ to model B. Such probabilities change according
to a combination of model fit/unfit (trial and error) and their shocks, and an
initial prior belief $p_o$.
The expected loss function for period $t$ becomes:
\begin{equation}
L_(p_\tau)_\tau = p_\tau L_{\tau|A} + (1 - p_\tau) L_{\tau|B}
\end{equation}
for $L_{\tau|A} $ and $ L_{\tau|B}$ representing the value functions conditional
on model A or B being the true model. This loss function is consistent with
microeconomic expected utility theory. 
\newline
\newline
The central bank problem without commitment under uncertainty is solved using a
similar method than the model certainty case. The objective is to solve for the
prior-conditional Lagrangian:
\begin{equation}
\mathcal{L}_\tau (p_\tau) = L_\tau (p_\tau) + \sum_{t=\tau}^\infty \lambda_t \beta^t E_t [(\pi_{t+1} + \gamma_y y_t + \varepsilon_\pi - \pi_t)|p_t]
\end{equation}
\end{frame}

\begin{frame}
The first order conditions with respect to $\{\pi_t\}^{\infty}_{t=\tau}$ and $\{y_t\}^{\infty}_{t=\tau}$
are basically the same as in the case with model certainty, that is, equations (13) and
(14), except for the fact that expectations take the prior in consideration.
Agents expectations and the Central Banker' expectations on inflation regarding
the prior are the same by assumption. Therefore, (15) becomes:
\begin{equation}
-\frac{\gamma}{w} E_t [\pi_t|p_t] = E_t [y_t|p_t].
\end{equation}
By applying the expectations operator conditional on the prior to the Phillips
curve, equation (16) becomes:
\begin{equation}
\left (1 + \frac{\gamma}{w} \right ) E_t (\pi_t | p_t) = \beta E_t (\pi_{t+1} |p_t) + \rho_\pi \varepsilon_{\pi,t-1}
\end{equation}
Dealing with $E_t (\pi_{t+1} |p_t)$ is troublesome because it is a conditional
expectation. The author takes a guess: a linear functional form in terms of the
previous shocks:
\begin{equation}
E_\tau (\pi_t|p_t) = \Lambda^{BAY} \rho_\pi \varepsilon_{\pi,t-1}.
\end{equation}
Parameter $\Lambda^{BAY}$ needs to be determined. The solution is the same as in
the case of model certainty:
\end{frame}

\begin{frame}
\begin{equation}
\Lambda^{BAY} = \frac{\rho_\pi}{1+\frac{\gamma^2}{w} - \beta \rho_\pi}
\end{equation}
\newline
\newline
Maybe this shows that regardless of model certainty or uncertainty, the Central
Bank goal is the same.  He will try to maintain inflation as if he knew the true
model, given what he knows. The difference will be the action he takes to
achieve the goal.
Because $\Lambda^{BAY} = \Lambda$, the law of motion for $\{E_{\tau}(\pi_t | p_t)\}^{\infty}_{t=\tau}$
is the same as the case of model certainty:
\begin{equation}
E_t[\pi_t|p_t] = \Lambda^{BAY} \rho_\pi^{t+1-r} \varepsilon_{\pi,\tau-1}
\end{equation}
Using equation (11) we can obtain the expected path for output:
\begin{equation}
E_\tau[y_t] = -\frac{\gamma_y}{w} E_\tau [\pi_t].
\end{equation}
Taking the Bayesian expectation in the aggregate demand equation (11) and
clearing out interest rate:
\end{frame}

\begin{frame}
\begin{equation}
i_\tau^{BAY} = \frac{1}{\Psi(p_t)} [E_\tau (y_{\tau+1} - y_\tau|p_t) + \chi E_\tau (\pi_{\tau+1}|p_t) + E_\tau (\mu_{y,\tau}|p_t)]
\end{equation}
where:
\begin{equation}
\Psi (p_t) = \chi - (p_t \theta^H + (1-p_t) \theta^L).
\end{equation}
Replacing the previous results, a similar result is found:
\begin{equation}
i_\tau^{BAY} = \frac{1}{\Psi(p_t)} \left [ \left ( \chi + \frac{\gamma_y}{w} ( \frac{1}{\rho_\pi} - 1) \right ) \Lambda \rho_\pi \varepsilon_{\pi,\tau-1} + E_\tau (\mu_{y,\tau}|p_t) \right ]
\end{equation}
When $\rho_\pi = 0$, 
\begin{equation}
i_t^{BAY} = \frac{1}{\Psi(p_t)} E_t[\mu_{y,t}|p_t].
\end{equation}
Given the data available at $\tau$, $E_t[\mu_{y,t}|p_t]$ can be computed by
taking the weighted average of the shocks affecting the model. Summarizing
results in a theorem, if we can call that a theorem,
\end{frame}

\begin{frame}
\begin{theorem}
 In the case of model uncertainty, assuming our knowledge assumptions, the
 Central Banker has the following optimal Bayesian monetary policy without
 commitment:
\begin{equation}
i_t^{BAY} (p_\tau) = \frac{1}{\Psi (p_\tau)} \left [ \left ( \chi_\pi + \frac{\gamma}{w} (\frac{1}{p_\pi} - 1 ) \right ) \Lambda \rho_\pi \varepsilon_{\pi,\tau-1} + E_\tau (\mu_{y,\tau}|p_\tau) \right ]
\end{equation}
where
\begin{equation}
\psi(p_\tau) = \chi - \left (p_t \theta^H + (1-p_t) \theta^L \right )
\end{equation}
and
\begin{equation}
\mu_{y,t} = \chi r_t^n - \theta i_t^* - \theta \varepsilon_{et} +
\varepsilon_{y,t}.
\end{equation}
\end{theorem}
\end{frame}

\begin{frame} 
\frametitle{State-Space representation} 
The exogenous state vector equation is:
\begin{equation}
S_t = A(p_t) S_{t-1} + B w_t
\end{equation}
In matrix notation:
\tiny{
\begin{equation}
\begin{bmatrix}
\varepsilon_{\pi,t} \\
\varepsilon_{\pi,t-1} \\
\varepsilon_{y,t} \\
\varepsilon_{y,t-1} \\
r_t^n \\
r_{t-1}^n \\
i_t^* \\
\varepsilon_{e,t} \\
1 \\
\tilde{\varepsilon}_{y,t} \\
\tilde{\varepsilon}_{y,t-1}
\end{bmatrix}
=
\begin{bmatrix}
\rho_\pi & 0 & 0 & 0 & 0 & 0 & 0 & 0 & 0 & 0 & 0 \\
1 & 0 & 0 & 0 & 0 & 0 & 0 & 0 & 0 & 0 & 0 \\
0 & 0 & \rho_y & 0 & 0 & 0 & 0 & 0 & 0 & 0 & 0 \\
0 & 0 & 1 & 0 & 0 & 0 & 0 & 0 & 0 & 0 & 0\\
0 & 0 & 0 & 0 & \rho_r & 0 & 0 & 0 & r^n & 0 & 0 \\
0 & 0 & 0 & 0 & 1 & 0 & 0 & 0 & 0 & 0 & 0 \\
0 & 0 & 0 & 0 & 0 & 0 & \rho_i & 0 & r^n & 0 & 0 \\
0 & 0 & 0 & 0 & 0 & 0 & 0 & \rho_e & 0 & 0 & 0 \\
0 & 0 & 0 & 0 & 0 & 0 & 0 & 0 & 1 & 0 & 0 \\
J_1 & 0 & J_2 & 0 & J_3 & 0 & J_4 & J_5 & 0 & J_6 & 0 \\
0 & 0 & 1 & 0 & 0 & 0 & 0 & 0 & 0 & 0 & 0
\end{bmatrix}
\begin{bmatrix}
\varepsilon_{\pi,t-1} \\
\varepsilon_{\pi,t-2} \\
\varepsilon_{y,t-1} \\
\varepsilon_{y,t-2} \\
r_{t-1}^n \\
r_{t-2}^n \\
i_{t-1}^* \\
\varepsilon_{e,t-1} \\
1 \\
\tilde{\varepsilon}_{y,t-1} \\
\tilde{\varepsilon}_{t_t-2}
\end{bmatrix}
+
\begin{bmatrix}
\sigma_\pi & 0 & 0 & 0 & 0 \\
0 & 0 & 0 & 0 & 0 \\
0 & \sigma_y & 0 & 0 & 0 \\
0 & 0 & 0 & 0 & 0 \\
0 & 0 & \sigma_r & 0 & 0 \\
0 & 0 & 0 & 0 & 0 \\
0 & 0 & 0 & \sigma_i & 0 \\
0 & 0 & 0 & 0 & \sigma_e \\
0 & 0 & 0 & 0 & 0 \\
J_7 & 0 & 0 & J_8 & J_9 \\
0 & 0 & 0 & 0 & 0
\end{bmatrix}
\begin{matrix}
\nu_{\pi,t} \\
\nu_{y,t} \\
\nu_{r,t} \\
\nu_{i,t} \\
\nu_{s,t} 
\end{matrix}
\end{equation}
}
\end{frame}

\begin{frame} 
\frametitle{State-Space representation}
To get the endogenous state block, we can use equation (68)  that describes the
optimal policy reaction function for model uncertainty, and replace the results
in equation (11) for each model. Let $\theta^T$ be the true model and $\theta^F$
the wrong mode:
\begin{equation}
y_t = \left ( \chi - \frac{\gamma}{w} \right ) \Lambda^{BAY} \rho_\pi \varepsilon_{\pi,t-1} - \left ( \chi - \theta^T \right ) i_t + \mu_{y,t}
\end{equation}
\begin{equation}
\tilde{y}_t = \left (\chi - \frac{\gamma}{w} \right ) \Lambda^{BAY} \rho_\pi \varepsilon_{\pi,t-1} - \left ( \chi - \theta^F \right ) i_t + \tilde{\mu}_{y,t}
\end{equation}
Subtract both equations and equate LHS to 0:
\begin{equation}
\mu_{y,t} + \left ( \theta^T - \theta^F \right ) i_t - \tilde{\mu}_{y,t} = 0
\end{equation}
Replace both terms by their respective definitions:
\begin{equation}
\varepsilon_{y,t} + \left ( \theta^T - \theta^F \right ) i_t = \tilde{\varepsilon}_{y,t}
\end{equation}
which implies
\begin{equation}
\rho \varepsilon_{y,t-1} + \nu_{y,t} + \left ( \theta^T - \theta^F \right) i_t = \rho_y \tilde{\varepsilon}_{y,t-1} + \tilde{\nu}_{y,t}
\end{equation}
\end{frame}

\begin{frame} 
\frametitle{State-Space representation}
$\varepsilon_{y,t}$ and $\tilde{\varepsilon}_{y,t}$ are state variables for the
Central Bank, which implies $\tilde{\nu}_{y,t}$ is endogenous. To obtain an
explicit solution, we use the linear form of the Central Bank's policy as a
function of the innovations to the system.
\newline
\newline
\textbf{Interest Rate:} The solution to the optimal policy is:
\begin{equation}
i_\tau^{BAY} = \frac{1}{\Psi (p_t)} \left [ \left ( \chi + \frac{\gamma_y}{w} ( \frac{1}{\rho_\pi} - 1) \right ) \Lambda \rho_\pi \varepsilon_{\tau-1} + E_\tau (\mu_{y,t}|p_t) \right ]
\end{equation}
Define the following auxiliary parameters:
\begin{equation}
F_1 (p_t) = \frac{1}{\Psi (p_t)} \left ( \frac{\gamma}{w} ( \frac{1}{\rho_\pi} - 1 ) - \chi \right ) \Lambda \rho_\pi
\end{equation}
\begin{equation}
F_2 (p_t) = \frac{1}{\Psi (p_t)} (p_t) \rho_{y,t}
\end{equation}
\begin{equation}
F_3 (p_t) = \frac{1}{\Psi (p_t)} \chi \rho_r
\end{equation}
\end{frame}

\begin{frame}
\frametitle{State-Space representation}
\begin{equation}
F_4 (p_t) = -\frac{1}{\Psi (p_t)} \left (p_t \theta^T + (1-p_t) \theta^F \right )
\end{equation}
\begin{equation}
F_5 (p_t) = -\frac{1}{\Psi (p_t)} \left (p_t \theta^T + (1-p_t) \theta^F \right )
\end{equation}
\begin{equation}
F_6 (p_t) = \frac{1}{\Psi (p_t)} ( 1 - p_t) \rho_{y,t}
\end{equation}
and:
\begin{equation}
i_t = F_1 \varepsilon_{\pi,t-1} + F_2 \varepsilon_{y,t-1} + F_3 \varepsilon_{r,t-1} + F_4 \varepsilon_{i,t} + F_5 \varepsilon_{e,t} + F_6 \tilde{\varepsilon}_{y,t-1}
\end{equation}
\end{frame}

\begin{frame}
\frametitle{State-Space representation}
We replace this in equation 79 to compute the value of $\{J_s\}^5_{s=1}$, by 
regrouping terms to obtain the value of $\tilde{v}_{y,t}$,
\small
\begin{align}
\tilde{v}_{y,t} &= \rho \varepsilon_{y,t-1} + v_{y,t} + \left ( \theta^T - \theta^F \right ) i_t - \rho_y \tilde{\varepsilon}_{y,t-1}  \\
&= \left (\theta^T - \theta^F \right) F_1 \varepsilon_{\pi,t-1} + \left( \left( \theta^T - \theta^F \right ) F_2 + \rho \right ) \varepsilon_{y,t-1} + \left ( \theta^T - \theta^F \right ) F_3 \varepsilon_{r,t-1} \\
&\qquad + \left (\theta^T - \theta^F \right ) F_4 \rho_i \varepsilon_{i,t-1} + \left (\theta^T - \theta^F \right ) F_5 \rho_e \varepsilon_{e,t-1} + \left( \left ( \theta^T - \theta^F \right ) F_6 - \rho_y \right ) \tilde{\varepsilon}_{y,t-1} \\
&\qquad + v_{y,t} + \left (\theta^T - \theta^F \right ) F_4 \nu_{i,t} + \left (\theta^T - \theta^F \right ) F_5 \nu_{e,t}
\end{align}
\end{frame}

\begin{frame}
\frametitle{State-Space representation}
Summarizing:
{\tiny
\begin{equation}
\tilde{\varepsilon}_{y,t} = J_1 \varepsilon_{\pi,t-1} + J_2 \varepsilon_{y,t-1} + J_3 \varepsilon_{r,t-1} + J_4 \varepsilon_{i,t-1} + J_5 \varepsilon_{e,t-1} + J_6 \tilde{\varepsilon}_{y,t-1} + J_7 v_{y,t} + J_8 \nu_{i,t} + J_8 \nu_{e,t}
}
\end{equation}
\normalsize
where the solution to:
{\small
\begin{equation}
J_1 = \left ( \theta^T - \theta^F \right ) F_1
\end{equation}
\begin{equation}
J_2 = \left (\left ( \theta^T - \theta^F \right ) F_2 + \rho_y \right )
\end{equation}
\begin{equation}
J_3 = \left ( \theta^T - \theta^F \right ) F_3
\end{equation}
\begin{equation}
J_4 = \left ( \theta^T - \theta^F \right ) F_4 \rho_i
\end{equation}
\begin{equation}
J_5 = \left ( \theta^T - \theta^F \right ) F_5 \rho_e
\end{equation}
\begin{equation}
J_6 = \left ( \theta^T - \theta^F \right ) F_6
\end{equation}
\begin{equation}
J_7 = 1
\end{equation}
\begin{equation}
J_8 = \left ( \theta^T - \theta^F \right ) F_4
\end{equation}
\begin{equation}
J_9 = \left ( \theta^T - \theta^F \right ) F_5
\end{equation}
}
\end{frame}

\begin{frame}
\frametitle{State-Space representation}
\textbf{Inflation Equation: }Using equation 29 to proceed in the same manner as
in the case of certainty. The set $\{G_s\}_{s=\pi,y,r,i,e}$ remains the same.
\newline
\newline
\textbf{Output Gap Equation: }From equation 28, we have:
\begin{equation}
y_t = \left ( \chi - \frac{\gamma}{w} \right ) \Lambda^{BAY} \rho_\pi \varepsilon_{\pi,t-1} - (\chi-\theta) i_t + \chi r_t^n - \theta i_t^* \theta \varepsilon_{et} + \varepsilon_{y,t}
\end{equation}
Using the following auxiliary variables:
\small
\begin{equation}
H_1 (p_t) = \left ( \chi - \frac{\gamma}{w} \right ) \Lambda^{BAY} \rho_\pi - (\chi-\theta^T) F_1 (p_t)
\end{equation}
\begin{equation}
H_2 (p_t) = 1
\end{equation}
\begin{equation}
H_3 (p_t) = - (\chi-\theta^T) F_2 (p_t)
\end{equation}
\begin{equation}
H_4 (p_t) = \chi
\end{equation}
\begin{equation}
H_5 (p_t) = - (\chi - \theta^T) F_3
\end{equation}
\begin{equation}
H_6 (p_t) = -\theta^T - (\chi - \theta^T) F_4
\end{equation}
\begin{equation}
H_7 (p_t) = -\theta^T - (\chi - \theta^T) F_5
\end{equation}
\end{frame}

\begin{frame}
\frametitle{State-Space representation}
\begin{equation}
H_8 (p_t) = - (\chi - \theta^T) F_6
\end{equation}
\newline
\newline
\textbf{Nominal Depreciation and Matrix form: } Remains the same. The system
takes the form:
\begin{equation}
Z_t = C (p_t) S_t
\end{equation}
The main difference is in the appearance of parameter $p_t$. Summarizing:
\tiny
\begin{equation}
\begin{bmatrix}
i_t \\
\pi_t \\ 
y_t \\
E_[\Delta s_{t+1}]
\end{bmatrix}
=
\begin{bmatrix}
0 & F_1 & 0 & F_2 & 0 & F_3 & F_4 & F_5 & 0 & 0 & F_6 \\
G_1 & G_2 & G_3 & G_4 & G_5 & G_6 & 0 & 0 & 0 & 0 & 0 \\
0 & H_1 & H_2 & H_3 & H_4 & H_5 & H_6 & H_7 & 0 & 0 & H_8 \\
0 & F_1 & 0 & F_2 & 0 & F_3 & F_4-1 & F_5-1 & 0 & 0 & 0
\end{bmatrix}
\begin{bmatrix}
\varepsilon_{\pi,t} \\
\varepsilon_{\pi,t-1} \\
\varepsilon_{y,t} \\
\varepsilon_{y,t-1} \\
r_t^n \\
r_{t-1}^n \\
i_t^* \\
\varepsilon_{e,t} \\
1 \\
\tilde{\varepsilon}_{y,t} \\
\tilde{\varepsilon}_{y,t-1}
\end{bmatrix}
\end{equation}
\end{frame}

\begin{frame}
\frametitle{State-Space representation}
\textbf{Conditional Moments and Loss Functions}
Unconditional moments can be obtained via numerical simulations of the above
system. Conditional moments can be obtained analytically using the matrix forms.
The summarized system is:
\begin{equation}
S_t = A (p_t) S_{t-1} + B w_t
\end{equation}
\begin{equation}
Z_t = C(p_t) S_t
\end{equation}
Bayes's updating rule is:
\begin{equation}
p_{t+1} = \frac{p_t P(Data_t|Data_{t-1},M^A)}{p_t P(Data_t|Data_{t-1},M^A) + (1-p_t) P(Data_t|Data_{t-1},M^B) }
\end{equation}
By Cogley et al. (2005), we know that $p_{t+1}$ is a martingale. Conditional on
the true model, $p_{t+1}$ is either a super-martingale or a sub-martingale,
depending on which is the true model.
\end{frame}

\begin{frame}
\frametitle{State-Space representation}
\textbf{Moments Conditional on the True Model and State} To get the first
moments, recall that conditional expectations over the exogenous states are:
\begin{equation}
E_t [S_{t+1}|S_t,p_{t+1}] = A(p_{t+1} S_t
\end{equation}
Substituting this into the observable yields:
\begin{equation}
E_t [Z_{t+1} | S_t,p_{t+1} ] = C (p_{t+1}) [A(p_{t+1})] S_t
\end{equation}
, where (for simplicity):
\begin{equation}
M (p_{t+1}) = C (p_{t+1}) [A (p_{t+1})]
\end{equation}
To obtain second moments, recall the equivalence:
\begin{equation}
S_{t+1} = A (p_{t+1}) S_t + B w_{t+1}
\end{equation}
and therefore:
\begin{equation}
Z_{t+1} = C (p_{t+1}) [A(p_{t+1}) S_t + B w_{t+1}]
\end{equation}
\end{frame}

\begin{frame}
\frametitle{State-Space representation} 
The loss at period $t+1$ is:
\begin{equation}
Z'_{t+1} W Z_{t+1}
\end{equation}
where
\begin{equation}
\begin{bmatrix}
0 & 0 & 0 & 0 \\
0 & 1 & 0 & 0 \\
0 & 0 & w & 0 \\
0 & 0 & 0 & 0
\end{bmatrix}
\end{equation}
Substituting in the components of $Z_{t+1}$,
\begin{equation}
[A ( p_{t+1}) S_t + B w_{t+1}]' C(p_{t+1})' W C (p_{t+1}) [A(p_{t+1}) S_t + B w_{t+1}]
\end{equation}
Taking expectations and defining it as a value $V$:
\small
\begin{equation}
V(S_t, p_{t+1}) = S'_t M (p_{t+1})' W M (p_{t+1}) S_t + E \left [w'_{t+1} B' C (p_{t+1})' W C (p_{t+1}) B w'_{t+1} \right ]
\end{equation}
\end{frame}

\begin{frame}
\frametitle{State-Space representation} 
To get value conditional on true model, use $p_{t+1}$ to refer to the
probability of the event that $A$ is the true model, given by:
\tiny
\begin{equation}
L_{t|A} (S_t,p_{t+1}) = V(S_t,p_{t+1}|A) + \beta E \left [(p_{t+1}) L_{t+1|A} (S_{t+1},p_{t+2}) + (1-p_{t+1}) L_{t+1|B} (S_{t+1},p_{t+2})|A \right ]
\end{equation}
\small and
\tiny
\begin{equation}
L_{t|B} (S_t,p_{t+1}) = V(S_t,p_{t+1}|B) + \beta E \left [(p_{t+1}) L_{t+1|A} (S_{t+1},p_{t+2}) + (1-p_{t+1}) L_{t+1|B} (S_{t+1},p_{t+2})|B \right ]
\end{equation}
\small
Standard methods of computation of value functions allow to obtain both value
functions.
\end{frame}

\begin{frame}
Recall:
\begin{theorem}
 In the case of model uncertainty, assuming our knowledge assumptions, the
 Central Banker has the following optimal Bayesian monetary policy without
 commitment:
\begin{equation}
i_t^{BAY} (p_\tau) = \frac{1}{\Psi (p_\tau)} \left [ \left ( \chi_\pi + \frac{\gamma}{w} (\frac{1}{p_\pi} - 1 ) \right ) \Lambda \rho_\pi \varepsilon_{\pi,\tau-1} + E_\tau (\mu_{y,\tau}|p_\tau) \right ]
\end{equation}
where
\begin{equation}
\psi(p_\tau) = \chi - \left (p_t \theta^H + (1-p_t) \theta^L \right )
\end{equation}
and
\begin{equation}
\mu_{y,t} = \chi r_t^n - \theta i_t^* - \theta \varepsilon_{et} +
\varepsilon_{y,t}.
\end{equation}
\end{theorem}
\end{frame}

\begin{frame}
\begin{enumerate} 
\item The optimal policy reacts to all of the shocks to the model by weighting
effects into the loss function.
\item Reaction to shocks depend on the sign of $\psi(p_\tau)$, which depends on
the prior.
\item Computing optimal policies require computing the demand shocks for both
model.
\item The update of model A probability in the Bayesian approach depends on
Bayes odd's ratio:
\end{enumerate}
\begin{equation}
p_t = P (M^A | Data) = \frac{P(Data|M^A) P(M^A)}{P(Data|M^B) P(M^B) + P(Data|M^A) P(M^A)}
\end{equation}
This ratio shows that the probability of choosing a given model as true will
depend on a weighted average of the likelihood of the data at period $t$, conditional to each model and data. 
\end{frame}

\begin{frame}
\frametitle{Calibration: } Designed so that the textbook model and balance sheet
model indicate the policy maker react to financial shocks in opposite
directions.
\newline
\newline
Canada is chosen as benchmark small open economy for model A. Calibration of
$\theta_A$ and the variance of the process was carried out to match the
HPS-statistic of Canada and volatility of devaluations and interest rates. To
parameterize model B, $\theta_A$  is chosen to match the HPS-statistic of Peru,
an example of a fear of floating economy.
\end{frame}

\begin{frame}
\includegraphics[width=0.99\textwidth]{001.pdf}
\end{frame}

\begin{frame}
\includegraphics[width=0.99\textwidth]{002.pdf}
\end{frame}

\begin{frame}
\includegraphics[width=0.99\textwidth]{003.pdf}
\end{frame}

\section{Robust Policy}
\begin{frame} 
\frametitle{Robust Policy} 
Assume that the Central Bank has a fear of misspecication of his prior probabilities and
hence, optimizes not in accordance to multiplier preferences instead of expected utility
theory. Central Banks can fear that their updating rules are misspecified and
don't fully trust Bayes rule.
\newline
\newline
By using robust policies, the Central Banker will use Bayes rule as a pivotal
mechanism to asses risks, but will minimize loss for a set of priors close to the one
computed by data and Bayes rule. The robust Central Banker problem is
\scriptsize
\begin{equation}
L(p_\tau)_\tau = \min_{\bar{p}_\tau} \max_{i_t} \left \{ \bar{p}_\tau \left [ - L_{\bar{p}_\tau|,A} + \Theta \log \left ( \frac{\bar{p}_\tau}{p_\tau} \right ) \right ] + (1 - \bar{p}_\tau) \left [- L_{\bar{p}_\tau|B} + \Theta \log \left ( \frac{1 - \bar{p}_\tau}{1 - p_\tau} \right ) \right ] \right \}
\end{equation}
\normalsize
where $\Theta$ is the parameter that characterizes robust preferences. The
higher this parameter, the more the Central Bank will trust its prior.
\end{frame}

\begin{frame} 
\frametitle{Robust Policy} 
where $p_{\tau}$ is the usual prior probability. The Central Banker acts like
there is a bad agent trying to distort the prior probabilities assigned to each
model. He is not entirely free as is constrained by $\Theta$.  High values of $\Theta$
allow the evil lower distortions. The game is played simultaneously, and the
minimizing agent will take sequences $\{i_t\}$ of policy decisions to
minimize the Welfare function. The FOC (sufficient) is:
\begin{equation}
\left [ -L_{\bar{p}_\tau|,A} + \Theta \log \left ( \frac{\bar{p}_\tau}{p_\tau} \right ) \right ] - \left [ -L_{\bar{p}_\tau|B} + \Theta \log \left ( \frac{1-\bar{p}_\tau}{1-p_\tau} \right ) \right] + \Theta = 0
\end{equation}
Regrouping, this yields:
\begin{equation}
\left ( \frac{\bar{p}_\tau}{1-\bar{p}_\tau} \right ) = \left ( \frac{p_\tau}{1-p_\tau} \right ) \exp \left [ \frac{L_{\bar{p}_\tau|,A} - L_{\bar{p}_\tau|B} }{\Theta} \right ]
\end{equation}
If the loss originated from model A is bigger than the loss caused
by model B, the Central Banker will act according to $\bar{p}_{\tau} > p_{\tau}$.
\end{frame}

\section{Model Dynamics and Learning}
\begin{frame} 
Speed of convergence of the prior probability departing from initial values for the prior and the model:
\includegraphics[width=0.99\textwidth]{004.pdf}
\end{frame}

\begin{frame} 
%\begin{quote} 
When the true model is A, beginning from a prior of 10 per cent the
expected waited time for converging to a prior probability of 90 per cent is slightly above
5 years, roughly, the time interval of each of our cross-country sub-samples. 
%\end{quote}
\newline
\textbf{Learning Under Robust Policies} Replication when robust policies are
implemented, based on a value $\Theta = 2$.
\end{frame} 

\begin{frame} 
Speed of convergence of the prior probability departing from initial values for the prior and the model:
\includegraphics[width=0.99\textwidth]{006.pdf}
\end{frame}

\end{document}
