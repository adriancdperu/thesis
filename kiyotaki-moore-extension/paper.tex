\documentclass[12pt]{article}%
\usepackage{amssymb}
\usepackage{amsfonts}
\usepackage{amsmath}
\usepackage[nohead]{geometry}
\usepackage[singlespacing]{setspace}
\usepackage[bottom]{footmisc}
%\usepackage{indentfirst}
\usepackage{endnotes}
\usepackage{graphicx}%
\usepackage{rotating}
\usepackage{tabularx}
\setcounter{MaxMatrixCols}{30}
\newtheorem{theorem}{Theorem}
\newtheorem{acknowledgement}{Acknowledgement}
\newtheorem{algorithm}[theorem]{Algorithm}
\newtheorem{axiom}[theorem]{Axiom}
\newtheorem{case}[theorem]{Case}
\newtheorem{claim}[theorem]{Claim}
\newtheorem{conclusion}[theorem]{Conclusion}
\newtheorem{condition}[theorem]{Condition}
\newtheorem{conjecture}[theorem]{Conjecture}
\newtheorem{corollary}[theorem]{Corollary}
\newtheorem{criterion}[theorem]{Criterion}
\newtheorem{definition}[theorem]{Definition}
\newtheorem{example}[theorem]{Example}
\newtheorem{exercise}[theorem]{Exercise}
\newtheorem{lemma}[theorem]{Lemma}
\newtheorem{notation}[theorem]{Notation}
\newtheorem{problem}[theorem]{Problem}
\newtheorem{proposition}{Proposition}
\newtheorem{remark}[theorem]{Remark}
\newtheorem{solution}[theorem]{Solution}
\newtheorem{summary}[theorem]{Summary}
\newenvironment{proof}[1][Proof]{\noindent\textbf{#1.} }{\ \rule{0.5em}{0.5em}}
\makeatletter
%\def\@biblabel#1{\hspace*{-\labelsep}}
\makeatother
\geometry{left=1in,right=1in,top=1.00in,bottom=1.0in}
\begin{document}

\title{Liquidity Crisis in Bonds and Equities Markets\thanks{The author would like to thank Koichi Futagami,
Takumi Motoyama, Kizuku Takao, Tomoya Kazumura, Takaaki Morimoto, Miho Sunaga, Sebastian Lesnic and Hiromasa Matsura for their help
and useful discussions. All remaining errors are the author's responsibility.}}
\author{Adrian Campos\thanks{Graduate School of Economics, Osaka University. E-mail: \textit{adrian@investometrica.com}}}

\maketitle

\sloppy%

\onehalfspacing

\textbf{Abstract:}
As an extension of Kiyotaki \& Moore (Kiyotaki and Moore, 2012), this paper
introduces a simple monetary economy with two types of agents, investors and workers. Investors are
randomly provided with a compelling opportunity to actively invest at a given 
period using capital and a production function, and
become entrepreneurs. The remaining passive investors can keep fiat money or choose
to invest in equities issued by previous entrepreneurs, their own equity or 
sovereign bonds issued by the local authority. In this context, we study the
effect of an exogenous change in the liquidity of our model's main assets: equities and bonds. 
The first part of our essay reviews the literature and introduces our motivation.
After introducing our model assumptions, we characterize equilibrium in our model.
We later proceed to study liquidity shocks in our simple economy, and compare a pure shock version
versus a deterministic version of liquidity, using impulse response functions
and calibration techniques.
\strut

\textbf{Keywords:} Macroeconomics, Finance.

\strut
\textbf{JEL classification:} E44, E50.

\strut

\pagebreak%

\nocite{kiyotaki1,alo1,gandolfo1,lucas1,kiyotaki2,salas1,holmstrom1,bigio1}

\section{Introduction}
Kiyotaki and Moore (Kiyotaki and Moore, 2002) have provided macroeconomists with
a powerful workhorse monetary model to study how asset prices and
aggregate economic activity fluctuate with recurrent shocks to productivity and
liquidity. The model depicts money as a high-liquidity asset inside a multi-asset economy\footnote{As a matter of fact, the only difference amongst assets is the degree of liquidity.}. Because of the fundamental role that liquidity has to differentiate amongst assets, the model has ben
labelled as a \emph{liquidity-in-advance} model, in contrast with the
traditional 
\emph{cash-in-advance} way of introducing  money into a neoclassical model (Lucas and Stokey,
1987).

With the notable exceptions of Bigio (2010), who develops the full stochastic version
of the model to study random liquidity shocks, and Salas (2013), which
uses the model to show that when credit constraints are high enough first-best allocations are not
reached, few extensions or dynamic analysis have been made out of Kiyotaki and Moore liquidity model.
To the best of our knowledge, no extension of this model has been carried out in
order to study liquidity crisis in the market of bonds. 

The original version of Kiyotaki and Moore describes an economy with the resale
of assets as main feature. There is a group of entrepreneurs that use their own
capital stock and skills to produce output from labor, which is supplied by
households. The investment technology needed for producing new capital from
previous output is not available to everybody. On the contrary, in each period
only some entrepreneurs are able to invest, as investment opportunities are
randomly distributed. For each period, there will a proportion of entrepreneurs
that invest, and a proportion that saves due to the lack of an investment
opportunity. Investing entrepreneurs need to acquire output for producing new
capital, and therefore they have a stimulus to issue equity claims to the
capital's future returns. Importantly, investors can only pledge a fraction
$\theta$ of the new capital's future returns. For $\theta$ low enough, he must
use his own resources to finance its investment costs.

In the original version, for a given period $t$ investors will have two kind of
assets in their balance sheets, which can be resold to raise funds: money $m_t$
and equity $n_{t}$, which has been previously issued by other entrepreneurs. In
consistency with vast empirical evidence, equity is assumed to be less liquid
than money. The degree to which equity is illiquid is parameterized by assuming in
each period agents can only sell a proportion $\phi$ of their holding equity.

In our extension, there is a third asset that we refer as bonds, which can be
thought as sovereign bonds issued by a given government to finance its
operations. Bonds pay interest, and therefore their future
value is higher than the present value under normal circumstances. Bonds can be
sold to investors, which now can choose to divide their portfolios at a given time $t$
between fiat money $m_t$, equity $n_{t}$ and sovereign debt represented in bond
$b_{t}$. We importantly parameterize the degree to which bonds are illiquid, by
assuming that in each period agents can only sell a proportion $\psi$ of their
holding bonds. Crucially, we suppose that equity is less liquid than bonds,
which are in turn less liquid than money: $\psi < \phi$.

Kiyotaki and Moore acknowledge that in practice there is a rich diversity of
assets, and thus there is also a wide spectrum of liquidity degrees: credit
default swaps, collateralized debt obligations and equity derivatives are some
exotic examples. Even inside the same category, there are important differences
in liquidity. For example, it is a well-known stylized fact that shares of
public companies are more liquid than shares in privately-held businesses, which
are usually traded over the counter. As such, Kiyotaki and Moore decide to
englobe all financial assets that are essentially as liquid as money under the
term ``money", and use the heading of ``equity" to refer to all financial assets
that are less than perfectly liquid.

We think that englobing sovereign bonds under the term ``money" is valid under
normal circumstances. For a given country, there is an active market of bonds
when the spread between the country's bond yield and a safe bond - usually an
American T-bill - is small. Under this context, sovereign bonds tend to be quite
liquid and thus may show money-like properties. However, it is also a well-known fact that
as the credit rate for a given country is degraded, the liquidity of its
sovereign debt decreases. In extreme cases, such as a default situation,
sovereign debt could be completely illiquid. Therefore, a theoretical study of liquidity shocks
could be more complete if the chosen model also describes the market for bonds.
We try to close this gap by introducing a new asset in Kiyotaki and Moore, representing the bonds asset class,
which is bought by investors due to its positive
expected yield, has its own market, and therefore, its own liquidity.

Our economy keeps most of the fundamental assumptions used in Kiyotaki and
Moore. New capital investment cannot be completely self-financed by issuing new equity because
there is a borrowing constraint represented by $\theta$. $\phi$ and $\psi$ are
resale-ability constraints, meaning that sufficient of the old equity or sovereign
debt cannot change hands quickly. Under this context, like in the original
model, fiat money can help lubricate the economy, for certain values of
$\theta$, $\psi$ and $\phi$. Whether or not agents use fiat money is determined
endogeonously. The fourth part of our paper is dedicated to study the conditions
by which our model becomes a monetary economy. We complement the analysis with
the usual analysis of equilibria conditions, steady states and dynamic system. 

Later, we proceed to study shocks to $\phi$ and $\psi$. Such
shocks were common during the Lehman Brothers financial turmoil, where not only equity of
several important corporations became illiquid in very short spans of time,
but sovereign debt of countries such as Iceland and Greece became worthless.
To use impulse-response reactions to analyze the reaction in the bond market to
a sudden change in liquidity for equities, or vice-versa, we decided to exogenously define a relation between $\phi$ and $\psi$ that is
consistent with empirical evidence. 

An exogenous definition of the relation between $\psi$ and $\phi$ could be, at first sight, problematic.
When Russia defaulted in 1998, its major stock index plummeted 65$\%$ in 36 trading
days. This fact suggests a positive correlation between liquidity fluctuations of bonds and
equities. This may occur due to the fact that investors are not willing to have
assets related in any way to a country where uncertainty is high, and where
economic growth for the next period is expected to be negative, an uncertainty
effect.

However, when Argentina defaulted in 2001, its major stock index, the Merval,
rose 130$\%$ two months later. This fact suggests a negative correlation between
liquidity fluctuation of bonds and equities. This may occur due to the fact that
there is a trade-off relation between investing in equities and investing in
bonds. If the uncertainty effect is not strong enough, investors may
decide to shift their investment in a given country from sovereign bonds to
equities, rather than leaving the country. Hence, there is also a trade-off
effect.

A complete endogenization of the relation described above goes beyond the scope of our paper,
because we believe that in order to model the interactions between uncertainty
effects and trade-off effects, it is not enough to define the maximization
problem of the government of a given country, but to stochastically model a global economy
where investors can choose to leave the country under certain conditions. Instead, we decide to describe the relation between liquidity of equities and
bonds exogenously.

In our framework, the difference between these two events is the magnitude of the liquidity
shock to the bonds market. Although Argentina 2001 and
Russia 1998 are situations were bonds markets became
highly illiquid, Argentina's bonds were relatively more liquid, because of
rumors that the government will engage into debt restructuring. Hedge funds with
tolerance to high risk saw some value in Argentinian bonds, even after the
government announced a USD 93 billion default on December 2001, because there was a
high possibility that the Argentinian government would announce a swap program, under
pressure from the International Monetary Fund, international courts and the Bush
administration. Indeed, in January 2005 the Argentine government offered the first
debt restructuring to affected bondholders, with nearly 76\%  of defaulted bonds
exchanged and brought out of default.

Therefore, we use a curve to describe the relation between liquidity of equities and
bonds. A sudden change in bonds liquidity $\psi$ inside the neighborhood between 
0 and $\overline{\psi}$ describes the Russian situation, whereas a sudden change in bonds liquidity
inside the neighborhood after $\overline{\psi}$ describes the Argentinian
situation. To capture this relation, let $\psi_t \equiv \lambda_{ \psi} b_t$
and  $\phi_t \equiv \lambda_{ \phi} n_t$ for $\lambda_{ \psi}$ and $\lambda_{ \phi}$ representing
constants, that is, assume that liquidity is
composed of a variable component (the quantity of bonds or equities in the economy)
and a constant component. This relation is shown in the next Figure. This is 
a function with a local maximum, where $\lim_{b \to 0} \psi = 0$, $\lim_{b \to 0} \phi =
0$ and $\lim_{b \to \infty} \phi = 0$:

\begin{figure}[h]
\caption{Liquidity of bonds and equities}
\centering
\includegraphics[width=0.73 \textwidth]{04.pdf} \\
\footnotesize
\label{longgdp}
\end{figure}

The merit of describing this relation between
the liquidity of bonds and equities is that we can now explain in a single graph
the case for Russia 1998 and Argentina 2001. With this in mind, we add an
exogenous relation between liquidity of bonds and equities that resembles our
curve, and use this relation to link markets of bonds and equities when we
introduce the bonds asset class in Kiyotaki and Moore.

In the next section, we take a look at Kiyotaki \& Moore, and introduce bonds 
and the relation between bonds and equities
liquidities described above. Section 3 describes equilibria. Section 4
includes a numerical analysis and study of shocks with perturbations techniques,
 and the last chapter presents our conclusions.

\newpage
\section{The model economy}
We consider an discrete-time, infinite-horizon economy. In our model, five objects are traded among agents: a
nondurable general output good, labour, equity, sovereign bonds and fiat money, where fiat money
is useless per se, and is assumed to be in fixed supply $M$. There are two populations of
agents, investors and workers, each with unit measure.
\subsection{Investors}
We assume that at date $t$, the representative investor has expected discount utility
\begin{equation}
\mathbb{E}_t [\sum^{\infty}_{s=t} \beta^{(s-t)}u(c_s)]
\end{equation}
of consumption path $\{c_t, c_{t+1}, c_{t+2}, \dots \}$. We assume a logarithmic
function for utility, $u(c) = log(c)$, and the usual constraint $0<\beta<1$.
Like in the original model, the representative investor has no labor endowment.
Instead, he has access to a constant-returns-to-scale technology for producing
nondurable general output using its capital $k_t$. It employes labor
$l_t$ to produce
\begin{equation}
y_t = A_t (k_t)^{\gamma}(l_t)^{1-\gamma},
\end{equation}
where $0<\gamma<1$. We also assume that $A_t>0$. $A_t$ is common to all investors, and
follows a stationary stochastic process. As usual, all production is completed within $t$ period.
We assume capital depreciation at a factor $\lambda k_t$ for $0<\lambda<1$. 

The representative investor employs labour with real wage rate $w_t$. As in Kiyotaki and Moore, 
gross profit is proportional to the capital stock. It depends upon output,
wages and capital in the following manner:
\begin{equation}
y_t - w_{t}k_t = r_{t}k_{t}.
\end{equation}

The investor may or may to have an opportunity to produce new capital stock, that is, to access
the constant-returns technology at each date $t$ with probability $\pi$. In the case the investor has an opportunity
to produce capital stock, he will obtain $i_t$ units of capital from $i_t$ units of
general output good. Also like in the original model, investment opportunities
are independently distributed across investors, and through time. Investment,
like production, is completed within each period $t$, but newly produced capital at $t$ does not
become available as production input until the next period, $t+1$:
\begin{equation}
k_{t+1} = \lambda k_t + i_t.
\end{equation}
To simplify the analysis, and make the distribution of capital and
asset holdings across investors well-behaved, it is assumed that there is no insurance market against
having an investment opportunity, and that the discount factor is larger than
the fraction of capital left after production:
\begin{equation}
\beta < \lambda.
\end{equation}

Investors with an opportunity to produce new capital stock can issue equity
claims to the future returns of newly produced capital. If equity is normalized
at one unit for date $t$, future returns are expressed as $r_{t+1}$ output for
date $t+1$, $\lambda r_{t+2}$ at date $t+2$, $\lambda^{2} r_{t+3}$ at date $t+3$
and so on.

There are two extremely important assumptions that Kiyotaki and Moore make.
First, investors who produce new capital can pledge at most $\theta$ fraction of
future returns from his new capital. This is a borrowing constraint. 

Second, investors cannot sell their equity holdings as quickly
as money. Before the investment opportunity disappears, he can only sell
fraction $\phi_t$ of his equity holdings within a period. Kiyotaki and Moore take
$\phi_t$ as an exogenous parameter of liquidity of the equity, and assume
aggregate productivity $A_t$ and the liquidity of equity $\phi_t$ follow a
stationary Markov process in the neighborhood of the constant unconditional
mean $(A, \phi).$ This is a liquidity constraint.

In our extension, we assume investors cannot sell their equity holdings, neither
their bond holdings as quickly as money. Within a period, the investor can only
sell a fraction $\phi_t$ of his equity holdings, and a fraction $\psi_t$ of his bonds
holdings. We assume aggregate productivity $A_t$, the liquidity of equity $\phi_t$
and the liquidity of bonds $\psi_t$ follow a stationary Markov process in the
neighborhood of the constant unconditional mean  $(A, \phi, \psi).$ 

Summarizing, in our framework, the investor has four types of assets  in his portfolio: fiat
money, equity of other investors, un-mortgaged capital stock, and sovereign bonds. 

\begin{center}
    \begin{tabular}{ | l | l | l | p{5cm} |}
    \hline
    Assets & Liabilities \\ \hline
    money & own equity issued \\ \hline
    equity of others & \\ \hline
    bonds & \\ \hline
    own capital stock & net worth \\
    \hline
    \end{tabular}
\end{center}

To simplify the analysis, it is assumed that at every period the investor can
remortgage up to a fraction $\psi_t$ of his un-mortgaged capital stock. This
causes equity of other investors and un-mortgaged capital stock to become perfect
substitutes since both pay the same return of $r_{t+1}$ at date $t+1$, $\lambda r_{t+2}$
at $t+2$,  $\lambda^2 r_{t+3}$ at $t+3$, and so on per unit. This reduces the
amount of assets to three: equity in a broad sense, fiat money, and sovereign bonds.
 
For $n_t$ the quantity of equity, $m_t$ the money and $b_t$ the amount of bonds
held by an individual investor at start of period $t$, the liquidity constraints
are expressed as:

\begin{equation}
n_{t+1} \geq (1-\theta)i_{t} + (1-\phi_t)\lambda n_t + (1-\psi_t)b_t,
\end{equation}
\begin{equation}
m_{t+1} \geq 0,
\end{equation}
\begin{equation}
b_{t+1} \geq 0.
\end{equation}

Intuitively, this shows that the investor who actively invests $i_t$ can issue
at most $\theta i_t$ equity, can resell at most $\phi_t$ fraction of equity
after depreciation and can resell at most $\psi_t$ fraction of bonds during this
period. Equation (6) implies that the investor's fiat money cannot be negative.
Equation (7) implies that the investor's amount of bond holdings cannot be
negative.

Finally, we can define a budget constraint for the investor agent.
For $q_t$ the price of equity in terms of general output, $p_t$ the price of
money in terms of general output, $z_t$ the price of bonds in terms of
general output, and $d_t$ the return of bonds at $t$ period,  the investor's flow of funds constraint for period $t$ is given
by
\begin{equation}
c_t + i_t +q_t(n_{t+1}-i_t-\lambda n_t)+p_t (m_{t+1}-m_t)+z_t(b_{t+1}-b_t) = r_t
n_t + d_t b_t.
\end{equation}

Intuitively, this tries to express that the investors's expenditure on
consumption, investment and net purchases of money, equities and bonds is equal
to his dividend income, which in turn is proportional to the holding of equity at the
start of this period.
\subsection{Workers}
At period $t$, the representative work has expected discount utility
\begin{equation}
\mathbb{E}_t [\sum^{\infty}_{s=t} \beta^{s-t} U[c'_s - \frac{\omega}{1+\nu}(l'_s)^{1+\nu}]
],
\end{equation}
of consumption path $\{ c_t,c_{t+1}, c_{t+2}, \dots \}$ given labor supply path $\{ l_t,
l_{t+1}, l_{t+2}, \dots \}$, for $\omega < 0$, $\nu > 0$ and $U[.]$ increasing
and strictly concave.

The flow of funds constraint of the worker is
\begin{equation}
c'_t +q_t(n'_{t+1}-i_t-\lambda n'_t)+p_t (m'_{t+1}-m'_t)+z_t(b'_{t+1}-b'_t) = w_t l'_t + r_t
n'_t + d_t b'_t.
\end{equation}
\begin{equation}
n'_{t+1} \geq 0,
\end{equation}
\begin{equation}
m'_{t+1} \geq 0,
\end{equation}
\begin{equation}
b'_{t+1} \geq 0.
\end{equation}
Notice that workers do not have investment opportunities. Furthermore, assume
they cannot borrow against their future labour income. 

\subsection{Government}
In our model, the government is the entity that issues sovereign bonds. We do not attempt to endogenously
explain government's behavior. The government faces the following budget constraint.
\begin{equation}
G_t + d_t B_t = p_t (M_{t+1}-M_t) + z_t (B_{t+1}-B_t),
\end{equation}
which can be expressed as
\begin{equation}
G_t +p_tM_t+z_t B_t + d_t B_t = p_t (M_{t+1}),
\end{equation}
or alternatively
\begin{equation}
G_t +p_tM_t+(1+\frac{d_t}{Z_t})Z_t B_t = p_t (M_{t+1}).
\end{equation}
Notice that $1+\frac{d_t}{Z_t}$ does not necessarily equal to the return of
equity. In other words, we do not assume a-priori the validity of no-arbitrage
condition in this model.
\section{Equilibrium}
\subsection{Case 1) \not \exists $\phi_t = f(\psi_t)$}
\subsubsection{Preliminaries}
In this subsection, we analyze our model assuming no particular relation between
 $\phi_t$ and $\psi_t$.
\begin{proposition}
(Labor market clearing condition) The labor demand for the representative investor
is:
\begin{equation}
k_t [ (1-\gamma)\frac{A_t}{w_t}]^{1-\gamma}.
\end{equation}
The labor aggregate supply for a unit-measure population of workers is:
\begin{equation}
[\frac{w_t}{\omega}]^{1/\nu}.
\end{equation}
\end{proposition}

\begin{proof}
We know that the typical worker has an expected discount utility\footnote{
As we notice later, when the utility function of the worker takes the particular
form:
\begin{equation}
\mathbb{E}_t \sum_{s=t}^{\infty} \beta^{s-t} [ln(c'_s) -
\frac{\omega}{1+\nu}l_s^{1+\nu}],
\end{equation}
characterizing a set of equilibrium solutions for the consumption, bond demand,
money demand and equities demand for not only the investor, but also for the
worker, becomes an easier task. Furthermore, in this particular case, the
labor-market clearing condition does not differ. However, since the utility
representation of Kiyotaki and Moore is more general, we keep the original assumption.
}
\begin{equation}
\mathbb{E}_t \sum_{s=t}^{\infty} \beta^{s-t} U[w_t l_s^{'}+r_t n'_t -q_t(n'_{t+1}-\lamda n'_t)-p_t (m'_{t+1}-m'_{t})
-\frac{\omega}{1+\nu}l'_{s}]^{(1+\nu)}.
\end{equation}
Expressing the first order condition:
\begin{equation}
\frac{ \partial \mathbb{E}_t \sum_{s=t}^{\infty} \beta^{s-t} U[w_t l_s^{'}+r_t n'_t -q_t(n'_{t+1}-\lamda n'_t)-p_t (m'_{t+1}-m'_{t})
-\frac{\omega}{1+\nu}l'_{s}]^{(1+\nu)}}{\partial l'_{s}} = 0,
\end{equation}
yields to
\begin{equation}
 l's = (\frac{w_t}{\omega})^{\frac{1}{\nu}}.
\end{equation}
Earlier, we assumed that $y_t = r_t k_t + w_t l_t$ and that $y_t = A_t (k_t)^{\gamma}
(l_t)^{(1-\gamma)}$. This can be re-expressed as:
\begin{equation}
A_t (k_t)^(\gamma) (l_t)^{1 -\gamma} = r_t k_t + w_t l_t.
\end{equation}
Re-arranging:
\begin{equation}
l_t = k_t [\frac{(1-\gamma)}{w_t}A_t]^{1/\gamma}.
\end{equation}
Finally, aggregating for all individuals:
\begin{equation}
l_d =  K_t [\frac{(1-\gamma)}{w_t}A_t]^{1/\gamma}. \qed
\end{equation}
\end{proof}
Now we know that the labor-market clearing condition requires
\begin{equation}
(\frac{w_t}{\omega})^{\frac{1}{\nu}}=  K_t [\frac{(1-\gamma)}{w_t}A_t]^{1/\gamma}.
\end{equation}
 
\begin{proposition} (Profit maximization) Maximized gross profit depends on
$K_t$, $A_t$, and parameters $\gamma$, $\nu$ and $\omega$. Furthermore, for the
representative investor, gross profit declines with $K_t$, but for the whole
sector, $r_t K_t$ increases with $K_t$.
\end{proposition}

\begin{proof}
We know that $(\frac{w_t}{\omega})^{\frac{1}{\nu}}=  K_t
[\frac{(1-\gamma)}{w_t}A_t]^{1/\gamma}$. Expressing $w_t$ in the right hand
side:
\begin{equation}
w_t^{\frac{1}{\nu}+\frac{1}{\gamma}} = \omega^{\frac{1}{\nu}} K_t
[(1-\gamma)A_t]^{1/\gamma}.
\end{equation}
Since $y_t = r_t k_t + w_t l_t$:
\begin{equation}
y_t - l_t[\omega^{\frac{1}{\nu}} K_t
[(1-\gamma)A_t]^{1/\gamma}]^{\frac{1}{\frac{1}{\nu}+\frac{1}{\gamma}}}=r_tk_t.
\end{equation}
Replacing the production function,
\begin{equation}
A_t(k_t)^{\gamma}(l_t)^{1-\gamma} - l_t[\omega^{\frac{1}{\nu}} K_t
[(1-\gamma)A_t]^{1/\gamma}]^{\frac{1}{\frac{1}{\nu}+\frac{1}{\gamma}}}=r_tk_t.
\end{equation}
This is equivalent to
\begin{equation}
r_t =A_t(k_t)^{\gamma -1}l_t^{1-\gamma}- l_t k_t^{-1}[\omega^{\frac{1}{\nu}} K_t
[(1-\gamma)A_t]^{1/\gamma}]^{\frac{1}{\frac{1}{\nu}+\frac{1}{\gamma}}}.
\end{equation}
By simple algebra,
\begin{equation}
r_t = \gamma (\frac{1-\gamma}{\omega})^{\frac{1-\gamma}{\gamma + \nu}} A_t^{\frac{1+\nu}{\gamma+\nu}}
K_t^{\frac{\gamma \nu -\nu}{\gamma+\nu}}.
\end{equation}
Let $a_t \equiv \gamma (\frac{1-\gamma}{\omega})^{\frac{1-\gamma}{\gamma + \nu}}
A_t^{\frac{1+\nu}{\gamma+\nu}}$ and $\alpha = \frac{\gamma (1+\nu)}{1+\nu}$.
Then, the above equation can be re-expressed as:
\begin{equation}
r_t = a_t (K_t)^{\alpha -1}.
\end{equation}
Since $\alpha \in [0,1]$, $\frac{\partial a_t(K_t)^{\alpha -1}}{\partial K_t}
< 0$ and $\frac{\partial K_t r_t}{\partial K_t} > 0$. \end{proof}
\begin{definition}
 (Stationary Recursive Competitive Equilibrium) A recursive competitive
 equilibrium is a  set of price functions $r*$, $q*$, $p*$, $z*$, $w*: \mathbb{S} \rightarrow
 \mathbb{R}_{+}$, allocation functions $n^j, c^j, l^j, i^j, b^j: \mathbb{S} \rightarrow
 \mathbb{R}_{+}$ for $j=s,i$, where $s$ denotes a 'saving' investor $(i=0)$ and $i$
 denotes an active investor $(i >0)$, such that:
  \begin{enumerate}
   \item Optimality policies solve the maximization problem of the
   representative investor (both when he saves and when he invests actively),
   representative work, and government.
    \item Goods markets clear at price $p*$.
     \item Labor markets clear at price $w*$.
      \item Equity markets clear at price $q*$.
       \item Bonds markets clear at price $z*$.
        \item Firms are run efficiently and per unit of capital profits equal
        $r*$.
\end{enumerate}
\end{definition}
Now we are able to start characterizing equilibrium. 
\subsubsection{A trivial case when money is value-less}
First, we analyze a trivial case, when $p_t=0$ and $q_t=1$.
We first assume that the constraints for investors $m_{t+1} >0$, $b_{t+1}>0$,
$n_{t+1}>0$, and constraints for workers $n'_{t+1}>0$, $m'_{t+1}>0$ and
$b'_{t+1}>0$ are not binding. Since the investor's flow of funds constraints is
$c_{t}+i_{t}+q_{t}(n_{t+1}-i_t-\lambda n_t) +p_t (m_{t+1}-m_t)+z_t (b_{t+1}-b_t)
= r_t n_t + d_t b_t$,
  \begin{enumerate}
   \item if $q_t >1$, investors would tend to allocate large quantities on $n$,
causing constraint $n_{t+1}>0$ to be binding. This leads to a contradiction.
   \item if $q_t <1$, investors would not allocate any resources to $n$, since
  _t (n_{t+1}-i_t-\lambda n_t) < 0$. 
\end{enumerate}
Therefore, in the not-binding case, $q_t = 1$ is needed for equilibrium. The
same logic applies to achieve $z_t = 1$. In this case, the budget constraint can
$c_t + i_t + (n_{t+1}-i_t-\lambda n_t) + (m_{t+1}-m_t)p_t + (b_{t+1}-b_t) = r_t
n_t + db_t$ can be reorganized to show consumption on the left hand side, as:
\begin{equation}
c_t = r_t n_t + db_t - i_t - (n_{t+1}-i_t -\lambda n_t) - (m_{t+1}-m_t)p_t -
(b_{t+1}-b_t).
\end{equation}
In this case, the Euler equation takes the form:
\begin{equation}
1 = \mathbb{E}_t [\beta \frac{c_t}{c_{t+1}}(r_{t+1}+\lambda)],
\end{equation}
for both the investor and worker. Notice that $p_t$ must equal zero. If $p_t >
0$, the Euler equation would show:
\begin{equation}
1 = \mathbb{E}_t [\beta \frac{c_t}{c_{t+1}}\frac{p_{t+1}}{p_t}],
\end{equation}
which would not satisfy in the neighborhood of the steady state. Thus, for this
trivial case, the budget constraint becomes
\begin{equation}
n_{t+1}-\lambda n_t = r_t n_t -c_t +d b_t -b_{t+1}+b_t.
\end{equation}
Since equation (6) can be written as
\begin{equation}
n_{t+1}-\lambda n_t \ge (1-\theta)i_t - \phi_t \lambda n_t + (1-\psi_t)b_t,
\end{equation}
the two equations above can be regrouped to form:
\begin{equation}
r_t n_t -c_t +d b_t -b_{t+1} \ge (1-\theta)i_t - \phi_t \lambda n_t - \psi_t b_t.
\end{equation}
As it can be seen from the shape of the budget constraint, money does not play
any value in the allocation of resources. Summarizing, for this trivial case, in
the neighborhood of the steady state:
  \begin{enumerate}
   \item Tobin's $q$ equals 1.
   \item $p_t = 0$.
   \item $z_t = 0$.
   \item The allocation of resources is first best.
\end{enumerate}
In this trivial case, the economy achieves first best allocation without money.
Therefore, money has no value.
\subsubsection{Characterization when $p_t >0$, $q_t > 0$, and $z_t > 0$}
Next, we move to the most interesting equilibrium case, where $p_t >0$, $q_t >
0$, and $z_t > 0$. We assume that the upper bounds on $\theta$, $\psi$ and
$\phi$ are tight enough. Let $V_t(m,n,b)$ be the value function of investors who
holds $(m_t, n_t, b_t)$ at the beginning of period $t$, before facing an
opportunity to actively invest with probability $\pi$. For the maximization problem of
the investor, the Bellman equation $V(m,n,b)$ can
be written as:
\begin{equation}
\underset{c^i_t +i_t+q_t (n^i_t -i_t -\lambda n^i_t)+p_t(m^i_{t+1}-m_t)
+z_t(b^i_{t+1}-b^i_t) = r_t n^i_t + d_t b^i_t.
}{\underset{m^i_{t+1} \ge 0,} {\underset{n^i_t \ge (1-\theta)i_t + (1-\phi_t)\lambda n^i_t +(1-\psi_t)b^i_t,}{\underset{s.t.}{\underset{ \{c^i_t, i^i_t, m^i_{t+1}, n^i_{t+1}, b^i_{t+1}\} } {\pi \text{Max} \{ln c^i_t + 
\beta \mathbb{E}_t [V_t (m^i_{t+1}, n^i_{t+1}, b^i_{t+1})]\}}}}}}
+
\underset{c^s_t +q_t (n^s_t -\lambda n^s_t)+p_t(m^s_{t+1}-m_t)
+z_t(b^s_{t+1}-b^s_t) = r_t n^s_t + d_t b^s_t.
}{\underset{m^s_{t+1} \ge 0,} {\underset{n^s_t \ge (1-\phi_t)\lambda n^s_t +(1-\psi_t)b^s_t,}{\underset{s.t.}{\underset{ \{c^s_t, m^s_{t+1}, n^s_{t+1}, b^s_{t+1}\} } {(1-\pi) \text{Max} \{ln c^s_t + 
\beta \mathbb{E}_t [V_t (m^s_{t+1}, n^s_{t+1}, b^s_{t+1})]\}}}}}}.
\end{equation}
Consider the case of an active investor, where the following constraints are
binding:
\begin{equation}
n^i_{t+1} = (1-\theta)i_t + (1-\phi_t)\lambda n^i_t + (1-\psi_t)b^i_t,
\end{equation}
\begin{equation}
m^i_{t+1} = 0.
\end{equation}
Then, unifying the budget constraint,
\begin{equation}
c^i_t+i+q_t(n^i_{t+1}-i_t-\lambda n^i_t)+z_t (b^i_{t+1}-b^i_t) = r_t n^i_t +d_t
b^i_t + p_t m_t,
\end{equation}
which can be re-expressed as
\begin{equation}
c^i_t + i_t(1-q_t)+q_t[(1-\theta)i_t +(1-\phi_t)\lambda n^i_t
+(1-\psi_t)b^i_t]+z_t(b^i_{t+1}-b^i_t) = n^i_t (r_t +q_t \lambda) + d_t b^i_t + p_t m_t, 
\end{equation}
leading to
\begin{equation}
c^i_t+i_t(1-\theta q_t) = n^i_t (r_t +q_t \lambda \phi_t)+p_t m_t + b^i_t [d_t +q_t (\psi_t
-1)+z_t]-b^i_{t+1}z_t,
\end{equation}
\begin{equation}
c^i_t+i_t(1-\theta q_t) + z_t (b^i_{t+1}-b^i_t)= (r_t + \lambda \phi_t q_t)n^i_t
+ p_t m_t + [d_t + (\psi_t -1)q_t]b^i_t.
\end{equation}
This equation equals consumption $c^i_t$, the downpayment for investment $i_t(1-\theta q_t)
$ and the downpayment for bond purchasing $z_t (b^i_{t+1}-b^i_t)$, to the
liquidity needs of the investor, expressed as $(r_t + \lambda \phi_t q_t)n^i_t
+ p_t m_t + [d_t + (\psi_t -1)q_t]b^i_t$.
We can use this budget constraint to re-express the Bellman equation as:
\begin{align*}
\V(m,n,b)= 

\underset{c^i_t +i_t+q_t (n^i_t -i_t -\lambda n^i_t)+p_t(m^i_{t+1}-m_t)
+z_t(b^i_{t+1}-b^i_t) = r_t n^i_t + d_t b^i_t.
}{\underset{m^i_{t+1} \ge 0,} {\underset{n^i_t \ge (1-\theta)i_t + (1-\phi_t)\lambda n^i_t +(1-\psi_t)b^i_t,}{\underset{s.t.}{\underset{ \{c^i_t, i^i_t, m^i_{t+1}, n^i_{t+1}, b^i_{t+1}\} } {\pi \text{Max} \{
ln[r_t n^i_t +\lambda \phi_t q_t n^i_t +p_t m_t + (d_t + (\psi_t -1)q_t)b^i_t -z_t(b^i_{t+1}-b^i_t)-i_t+\theta q_t i_t]
+\beta \mathbb{E}_t [V^1_t]\}}}}}}
\\
    &
    
\underset{c^s_t +q_t (n^s_t -\lambda n^s_t)+p_t(m^s_{t+1}-m_t)
+z_t(b^s_{t+1}-b^s_t) = r_t n^s_t + d_t b^s_t.
}{\underset{m^s_{t+1} \ge 0,} {\underset{n^s_t \ge (1-\phi_t)\lambda n^s_t +(1-\psi_t)b^s_t,}{\underset{s.t.}{\underset{ \{c^s_t, m^s_{t+1}, n^s_{t+1}, b^s_{t+1}\} } {+(1-\pi) \text{Max} \{
ln [r_t n^s_t + \lambda \phi_t q_t n^s_t +p_t m_t + (d_t + (\psi_t-1)q_t)b^s_t-z_t(b^s_{t+1}-b^s_t)]
+\beta \mathbb{E}_t [V^2_t]\}}}}}},

\end{align*}
where $V^1_t \equiv V(m^i_{t+1},n^i_{t+1},b^i_{t+1})$ and $V^2_t \equiv V(m^s_{t+1}, n^s_{t+1},
b^s_{t+1})$. Now, we proceed to calculate the first order conditions of $i_t,
$m^i_{t+1}$, $n^i_{t+1}$, $b^i_{t+1}$, $m^s_{t+1}$, $n^s_{t+1}$, and
$b^s_{t+1}$. 

For $b^i_{t+1}$:
\begin{equation}
\frac{\partial V(m,n,b)}{\partial b^i_{t+1}} = \frac{-z_t}{c^i_t}+\beta \frac{\mathbb{E}\partial V(m^i_{t+1},n^i_{t+1}
, b^i_{t+1})}{\partial b^i_{t+1}}=0.
\end{equation}
Notice that, by the Envelope theorem,
\begin{equation}
\frac{\partial V(m^i_{t+1},n^i_{t+1},b^i_{t+1})}{\partial b^i_{t+1}} = \frac{\partial
u(c_{t+1})}{\partial b^i_{t+1}}\bigg|_{m_{t+1}=m^i_{t+1},
n_{t+1}=n^i_{t+1},b_{t+1}=b^i_{t+1}},
\end{equation}
\begin{equation}
\frac{\partial V(m^i_{t+1}, n^i_{t+1}, b^i_{t+1})}{\partial b^i_{t+1}} = \frac{z_{t+1}+(d_{t+1}+\psi_{t+1}q_{t+1}-q_{t+1})}{c^i_{t+1}}
= \frac{z_{t+1}+[d_{t+1}+q_{t+1}(\psi_{t+1}-1)]}{c_{t+1}}.
\end{equation}
Therefore,
\begin{equation}
\frac{\partial V(m,n,b)}{\partial b^i_{t+1}} = -\frac{z_{t}}{c^i_t}+\beta \pi [\frac{z_{t+1}+[d_{t+1}+q_{t+1}
(\psi_{t+1}-1)]}{c^{i,i}_{t+1}}]+\beta (1-\pi)
\frac{z_{t+1}+d_{t+1}+q_{t+1}(\psi_{t+1}-1)}{c^{i,s}_{t+1}}=0.
\end{equation}

For $n^i_{t+1}$:
\begin{equation}
\frac{\partial V(m,n,b)}{\partial n^i_{t+1}}=\beta \mathbb{E}[\frac{\partial V(m^i_{t+1},n^i_{t+1}, b^i_{t+1})}{\partial n^i_{t+1}}]
= 0,
\end{equation}
\begin{equation}
\frac{\partial V(m^i_{t+1},n^i_{t+1},b^i_{t+1})}{\partial n^i_{t+1}}=
\frac{\partial u(c_{t+1})}{\partial n^i_{t+1}}
\bigg|_{m_{t+1}=m^i_{t+1},
n_{t+1}=n^i_{t+1},b_{t+1}=b^i_{t+1}},
\end{equation}
\begin{equation}
\frac{\partial V(m^i_{t+1},n^i_{t+1},b^i_{t+1})}{\partial n^i_{t+1}}=
\frac{(r_{t+1}+\lambda \phi_{t+1}q_{t+1})}{c^{i,i}_{t+1}},
\end{equation}
for both the case when investor has active opportunity to invest with
probability $\pi$, and has to save with probability $1-\pi$. Therefore,
\begin{equation}
\frac{\partial V(m,n,b)}{\partial n^i_{t+1}}=
\pi \beta [\frac{(r_{t+1}+\pi \phi_{t+1} q_{t+1})}{c^{i,i}_{t+1}}]
+ \beta (1-\pi) \frac{(r_{t+1}+\lambda \phi_{t+1}q_{t+1})}{c^{i,s}_{t+1}}=0.
\end{equation}

For $m^i_{t+1}$:
\begin{equation}
\frac{\partial V(m,n,b)}{\partial m^i_{t+1}}=\beta \mathbb{E}[\frac{\partial V(m^i_{t+1},n^i_{t+1}, b^i_{t+1})}{\partial m^i_{t+1}}]
= 0,
\end{equation}
\begin{equation}
\frac{\partial V(m^i_{t+1},n^i_{t+1},b^i_{t+1})}{\partial m^i_{t+1}}=
\frac{\partial u(c_{t+1})}{\partial m^i_{t+1}}
\bigg|_{m_{t+1}=m^i_{t+1},
n_{t+1}=n^i_{t+1},b_{t+1}=b^i_{t+1}},
\end{equation}
\begin{equation}
\frac{\partial V(m^i_{t+1},n^i_{t+1},b^i_{t+1})}{\partial m^i_{t+1}}=
\frac{p_{t+1}}{c^i_{t+1}},
\end{equation}
also for both the case when investor has active opportunity to invest with
probability $\pi$, and has to save with probability $1-\pi$. Therefore,
\begin{equation}
\frac{\partial V(m,n,b)}{\partial m^i_{t+1}}=
\pi \beta [\frac{p_{t+1}}{c^{i,i}_{t+1}}]+(1-\pi)\beta
[\frac{p_{t+1}}{c^{i,s}_{t+1}}]=0.
\end{equation}

For $i_t$, the derivation of the first order condition is pretty straight
forward,
\begin{equation}
\frac{\partial V(m,n,b)}{\partial i_t}=
-\pi \frac{(1-\theta q_t)}{c^i_t} = 0.
\end{equation}

The first order condition of $b^s_{t+1}$ is:
\begin{equation}
\frac{\partial V(m,n,b)}{\partial b^s_{t+1}}=-\frac{z_t}{c^s_t}+\beta \frac{\mathbb{E}\partial 
V(m^s_{t+1}, n^s_{t+1}, b^s_{t+1})}{\partial b^s_{t+1}}=0,
\end{equation}

\begin{equation}
\frac{\partial V(m^s_{t+1},n^s_{t+1},b^s_{t+1})}{\partial b^s_{t+1}}=
\frac{\partial u(c_{t+1})}{\partial b^s_{t+1}}
\bigg|_{m_{t+1}=m^s_{t+1},
n_{t+1}=n^s_{t+1},b_{t+1}=b^s_{t+1}},
\end{equation}

\begin{equation}
\frac{\partial V(m^s_{t+1},n^s_{t+1},b^s_{t+1})}{\partial b^s_{t+1}}=
\frac{-z_{t+1}+(d_{t+1}+(\psi_{t+1}-1)q_{t+1})}{c^s_{t+1}}
= \frac{z_{t+1}+[d_{t+1}+q_{t+1}(\psi_{t+1}-1)]}{c^s_{t+1}}.
\end{equation}

Therefore,
\begin{equation}
\frac{\partial V(m,n,b)}{\partial b^s_{t+1}}=-\frac{z_t}{c^s_t}
+(1-\pi)\beta \Big\{ \frac{z_{t+1}+[d_{t+1}+q_{t+1}(\psi_{t+1}-1)]}{c^{s,s}_{t+1}}
\Big\} +\pi \beta \Big\{ \frac{z_{t+1}+[d_{t+1}+q_{t+1}(\psi_{t+1}-1)]}{c^{s,i}_{t+1}}\Big\}.
\end{equation}

Now, the first order condition of $n^s_{t+1}$ is 
\begin{equation}
\frac{\partial V(m,n,b)}{\partial n^s_{t+1}}=
(1-\pi)\frac{\beta \mathbb{E}[\partial V(m^s_{t+1}, n^s_{t+1}, b^s_{t+1})]}{\partial n^s_{t+1}}
=0,
\end{equation}

\begin{equation}
\frac{\partial V(m^s_{t+1},n^s_{t+1},b^s_{t+1})}{\partial n^s_{t+1}}=
\frac{\partial u(c_{t+1})}{\partial n^s_{t+1}}
\bigg|_{m_{t+1}=m^s_{t+1},
n_{t+1}=n^s_{t+1},b_{t+1}=b^s_{t+1}},
\end{equation}

\begin{equation}
\frac{\partial V(m^s_{t+1},n^s_{t+1},b^s_{t+1})}{\partial n^s_{t+1}}=
\beta (1-\pi) \frac{\mathbb{E}\partial V(m^s_{t+1}, n^s_{t+1}, b^s_{t+1})}{\partial n^s_{t+1}}
= 0,
\end{equation}

\begin{equation}
\frac{\partial V(m,n,b)}{\partial n^s_{t+1}}
= \beta (1-\pi) \Big\{ \frac{(r_{t+1}+\lambda
\phi_{t+1}q_{t+1})}{c^{s,s}_{t+1}}\Big\}
+\beta \pi \Big\{ \frac{(r_{t+1}+\lambda \phi_{t+1}q_{t+1})}{c^{s,i}_{t+1}}
\Big\} = 0.
\end{equation}

Finally, the first order condition of $m^s_{t+1}$:
\begin{equation}
\frac{\partial V(m,n,b)}{\partial m^s_{t+1}}
= (1-\pi)\frac{\beta \mathbb{E}\partial V(m^s_{t+1}, n^s_{t+1}, b^s_{t+1})}{\partial m^s_{t;1}}
= 0,
\end{equation}

\begin{equation}
\frac{\partial V(m^s_{t+1},n^s_{t+1},b^s_{t+1})}{\partial m^s_{t+1}}=
\frac{\partial u(c_{t+1})}{\partial m^s_{t+1}}
\bigg|_{m_{t+1}=m^s_{t+1},
n_{t+1}=n^s_{t+1},b_{t+1}=b^s_{t+1}},
\end{equation}

\begin{equation}
\frac{\partial V(m^s_{t+1},n^s_{t+1},b^s_{t+1})}{\partial m^s_{t+1}}=
\frac{p_{t+1}}{c_{t+1}},
\end{equation}
for $\pi$ and $(1-\pi)$. Therefore:

\begin{equation}
\frac{\partial V(m^s_{t+1},n^s_{t+1},b^s_{t+1})}{\partial m^s_{t+1}}=
\pi \beta [\frac{p_{t+1}}{c^{s,i}_{t+1}}]+(1-\pi)
 \beta [\frac{p_{t+1}}{c^{s,s}_{t+1}}] = 0.
\end{equation}

Summarizing,
\begin{proposition}
For a binding case, equilibrium for the investor is characterized
by the following first order conditions:
\begin{equation}
\frac{\partial V(m,n,b)}{\partial b^i_{t+1}} = -\frac{z_{t}}{c^i_t}+\beta \pi \Big\{\frac{z_{t+1}+[d_{t+1}+q_{t+1}
(\psi_{t+1}-1)]}{c^{i,i}_{t+1}}\Big\}+\beta (1-\pi)
\Big\{\frac{z_{t+1}+d_{t+1}+q_{t+1}(\psi_{t+1}-1)}{c^{i,s}_{t+1}}
\Big\}=0.
\end{equation}

\begin{equation}
\frac{\partial V(m,n,b)}{\partial b^s_{t+1}}=-\frac{z_t}{c^s_t}
+(1-\pi)\beta \Big\{ \frac{z_{t+1}+[d_{t+1}+q_{t+1}(\psi_{t+1}-1)]}{c^{s,s}_{t+1}}
\Big\} +\pi \beta \Big\{ \frac{z_{t+1}+[d_{t+1}+q_{t+1}(\psi_{t+1}-1)]}{c^{s,i}_{t+1}}\Big\}
= 0.
\end{equation}

\begin{equation}
\frac{\partial V(m,n,b)}{\partial n^i_{t+1}}=
\pi \beta \Big\{ \frac{(r_{t+1}+\pi \phi_{t+1} q_{t+1})}{c^{i,i}_{t+1}}\Big\}
+ \beta (1-\pi) \Big\{ \frac{(r_{t+1}+\lambda \phi_{t+1}q_{t+1})}{c^{i,s}_{t+1}}\Big\}=0.
\end{equation}

\begin{equation}
\frac{\partial V(m,n,b)}{\partial n^s_{t+1}}
= \beta (1-\pi) \Big\{ \frac{(r_{t+1}+\lambda
\phi_{t+1}q_{t+1})}{c^{s,s}_{t+1}}\Big\}
+\beta \pi \Big\{ \frac{(r_{t+1}+\lambda \phi_{t+1}q_{t+1})}{c^{s,i}_{t+1}}
\Big\} = 0.
\end{equation}

\begin{equation}
\frac{\partial V(m,n,b)}{\partial m^i_{t+1}}=
\pi \beta [\frac{p_{t+1}}{c^{i,i}_{t+1}}]+(1-\pi)\beta
[\frac{p_{t+1}}{c^{i,s}_{t+1}}]=0.
\end{equation}

\begin{equation}
\frac{\partial V(m^s_{t+1},n^s_{t+1},b^s_{t+1})}{\partial m^s_{t+1}}=
\pi \beta [\frac{p_{t+1}}{c^{s,i}_{t+1}}]+(1-\pi)
 \beta [\frac{p_{t+1}}{c^{s,s}_{t+1}}] = 0.
\end{equation}

\begin{equation}
\frac{\partial V(m,n,b)}{\partial i_t}=
-\pi \frac{(1-\theta q_t)}{c^i_t} = 0.
\end{equation}
\end{proposition}


\subsubsection{Net worth of active investors}
Since constraint $n_{t+1} = (1-\theta)i_t+(1-\phi_t)\lambda n_t
+(1-\psi_t)b_t$, 
\begin{equation}
c^i_t + \frac{(1-\theta q_t)}{(1-\theta)}
[n^i_{t+1}-(1-\phi_t)\lambda n_t -(1-\psi_t)b_t]+z_t(b^i_{t+1}-b_t)
= (r_t+\lambda \phi_t q_t)n_t + p_t m_t +[d_t + (\psi_t -1)q_t]b_t,
\end{equation}
can be re-expressed as

\begin{equation}
  \begin{split}
 &c^i_t + \frac{(1-\theta q_t)n^i_{t+1}}{(1-\theta)}+z_t b^i_{t+1} =
\\ &[r_t+\lambda \phi_t q_t + \frac{(1-\phi_t)\lambda (1-\theta
q_t)}{(1-\theta)}]n_t+p_t m_t +
[d_t +(\phi_t -1)q_t +\frac{(1-\psi_t)(1-\theta q_t)}{(1-\theta)}+z_t]b_t.
\end{split} 
\end{equation}
Let $q^R_t \equiv \frac{1-\theta q_t}{(1-\theta)}$. Notice $q_t^R >1$, since $q_t >1$. Then, the
above expression can be re-expressed as
\begin{equation}
  \begin{split}
 &c^i_t + q^R_t n^i_{t+1}+z_t b^i_{t+1}=
\\ &r_t n_t +[\phi_t q_t +(1-\phi_t)q_t^R]\lambda n_t +p_t m_t
+[d_t +(\psi_t -1)q_t +(1-\psi_t)q^R_t+z_t]b_t.
\end{split} 
\end{equation}
The above equation shows the net worth of the active investor. $q^R_t$ can be
understood as the effective replacement cost of equity. In this case, the net
worth is equal to the sum of gross dividends, the value of depreciated equity,
and the value of bonds after accounting for the positive $d_t$ on their face
value, $(1-\psi_t)$ liquidity effect, and the effect from investing partially in
equity.

\subsubsection{Marginal utility of consumption}
Since we assumed logarithmic preferences, an investor saves fraction $\beta$
of its net worth, and spends fraction $(1-\beta)$:
\begin{equation}
  \begin{split}
 &c^i_t =
\\ &(1-\beta) \Big\{ r_t n_t + [\phi_t q_t + (1-\phi_t)q^R_t]\lambda n_t +p_t
m_t + [d_t+(\psi_t -1)q_t +(1-\psi_t)q^R_t+z_t]b_t \Big\}.
\end{split} 
\end{equation}
We recall that investment $i_t$ is equal to:
\begin{equation}
i_t = \frac{(r_t + \lambda \phi_t q_t)n^i_t + p_t m_t + [d_t + (\psi_t -1)q_t]b^i_t
+z_t (b^i_t -b^i_{t+1})-c_i
}{(1-\theta q_t)},
\end{equation}
where $(r_t + \lambda \phi_t q_t)n^i_t $ represents the downpayment per unit of
investment in equities, [$d_t + (\psi_t -1)q_t]b^i_t$ is the downpayment for
bonds, and $z_t (b^i_t -b^i_{t+1})-c_i$ represents the remaining liquidity. 
For the case of an investor that is not active, that is, for a saver, the budget
constraint is:
\begin{equation}
c^s_t +q_t n^s_{t+1} +p_t m^s_{t+1} +z_t(b^s_{t+1}) =r_t n_t +q_t \lambda n_t +
p_t m_t + z_t b_t.
\end{equation}
If the usual constraints $n_{t+1} \ge (1-\theta)i_t + (1-\phi_t)\lambda n_t
+ (1-\psi_t)b_t$ and $m_{t+1}\ge 0$ do not bind, the right hand side of the
above equation represents the passive investor's net worth. When compared
against the active investor's net worth, notice the passive investor has his
depreciated equity valued at $q_t$. The passive investor consumes:
\begin{equation}
c^s_t = (1-\beta)(r_t n_t + q_t \lambda n_t + p_t m_t + z_t b_t).
\end{equation}
Now that we described consumption for the 'active investing case' and the
opposite case, we can express marginal utility of consumption, which comes from
sacrificing one unit of $c_t$ for $\frac{1}{p_t}$ units of money,
$\frac{1}{q_t}$ units of equity or $\frac{1}{z_t}$ units of bonds:
\begin{equation}
u'(c_t) = \mathbb{E}_t \Big\{ \frac{p_{t+1}}{p_t} [(1-\pi)u'(c^s_{t+1})+\pi u'(c^i_{t+1})]
\Big\},
\end{equation}
\begin{equation}
u'(c_t) = (1-\pi) \mathbb{E}_t \Big\{ \frac{r_{t+1}+\lambda q_{t+1}}{q_t} u'(c^s_{t+1})
\Big\} +\pi \mathbb{E}_t \Big\{ \frac{r_{t+1}+\lambda \phi_{t+1}q_{t+1}+\lambda (1-\phi_{t+1})q^R
_{t+1}u'(c^i_{t+1})}{q_t}\Big\},
\end{equation}
\begin{equation}
u'(c_t) = (1-\pi) \mathbb{E}_t \Big\{ \frac{z_{t+1}}{z_t}u'(c^s_{t+1})
\Big\} +\pi \mathbb{E}_t \Big\{ 
\frac{[d_{t+1}+(\psi_{t+1}-1)q_{t+1}]+z_{t+1} u'(c^i_{t+1})}{z_t}
\Big\}.
\end{equation}
\subsubsection{Agregation}
TBC
\subsection{Case 2) \exists  $\phi_t = f(\psi_t)$}
In this part, we simplify the equilibrium characterization by assuming there
exists a relation between $\phi_t$ and $\psi_t$, which is defined a priori as:

\begin{equation}
\phi_t=\psi_t^{\overline{\alpha}} (\overline{x}-\psi_t)^{\overline{\beta}},
\end{equation}

defined over $[0,\overline{x}]$, with strictly positive $\overline{x}$, $\overline{\alpha}$ and $\overline{\beta}$. The idea
is to have a parametric, concave relation that resembles Figure 1. 


TBC

\section{Numerical analysis}
In this section we use numerical methods to obtain a solution for the
equilibrium conditions in the previous section. After we obtain a solution, we
use perturbations to study shocks to liquidity of equities and bonds. 

In particular, we compare two cases.

\subsection{Pure AR(1) case}
In the first case, we
suppose, as in Kiyotaki and Moore, that there is a law of motion for
productivity and liquidity $(A_t, \phi_t, \psi_t)$ which follows independent AR(1)
processes so that $a_t = \gamma (\frac{1-\gamma}{\omega})^{\frac{1-\gamma}{\gamma + \nu}} A_t^{\frac{1+\nu}{\gamma+\nu}}
$ and $\phi_t$ follow AR(1) as:

\begin{equation}
a_t-a = \rho_a (a_{t-1}-a) + \epsilon_{at},
\end{equation}
\begin{equation}
\phi_t-\phi=\rho_{\phi} (\phi_{t-1}-\phi) + \epsilon_{\phi t},
\end{equation}
\begin{equation}
\psi_t-\psi=\rho_{\psi} (\psi_{t-1}-\psi) + \epsilon_{\psi t},
\end{equation}

where $\rho_a$, $\rho_{\phi}$, $\rho_{\psi}$ \in (0,1)$, and for calibration purposes, $\rho_a$, $\rho_{\phi}$
and $\rho_{\psi}$
are set to $0.95$, $\epsilon_{at}$, $\epsilon_{\psi t}$ and $\epsilon_{\phi t}$ are
identically and independently distributed innovations to the levels of
productivity, liquidity of equities, and liquidity of bonds. 

\subsection{Semi-deterministic case}
In this case we assume
\begin{equation}
a_t-a = \rho_a (a_{t-1}-a) + \epsilon_{at},
\end{equation}
and this holds,
\begin{equation}
\phi_t=\psi_t^{\overline{\alpha}} (\overline{x}-\psi_t)^{\overline{\beta}},
\end{equation}
where $\rho_a$ is set to $0.95$, $\overline{x}$ is set to $1$, and $\overline{\alpha}$ and
$\overline{\beta}$ are set to $0.5$. $\epsilon_{at}$ represents an 
identically and independently distributed innovation to the level of
productivity.

For both cases we follow Del Negro et. al. (2011) for choosing parameters:
periods are quarters, $\pi = 0.05$, $\theta = 0.19$, $\psi = 0.19$, 
$\gamma = 0.4$, $\nu = 1$, $\beta = 0.99$, $\lambda = 0.975$, and $\phi = 0.19$.

\subsection{Comparison of impulse response functions}
TBA
\section{Concluding remarks}
TBA
\pagebreak%
\bibliographystyle{abbrv}
\bibliography{references}
\end{document} 
